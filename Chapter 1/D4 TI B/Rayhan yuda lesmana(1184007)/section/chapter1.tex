\chapter*{SEJARAH PYTHON}

\par
	Python dibuat oleh Guido Van Rossum pada awal tahun 1990-an. Bahasa python ini terinspirasi dari bahasa pemrograman ABC. Di tahun 1995, Guido membuat Corporation for National Research Initiative(CNRI)di Virginia Amerika, dan merilis beberapa versi python. Pada mei 2000, tim python pindah ke BeOpen.com dan membentuk tim BeOpen PythonLabs. Pada tahun 2001, dibentuk Organisasi nirlaba khusus untuk semual hal yang berkaitan dengan intelektual Python. Semua versi python bersifat open source.
		
\begin{enumerate}
\item Pyton 1.0 – 1.6 Rilis dari tahun 1994- pertengahan 2000
\item Pyton 2.0 – 2.7 Rilis dari tahun pertengahan 2000-2010 
\item Pyton 3.0 – 3.7 Rilis dari tahun 2008-2018
\end{enumerate}

\par
	Python yang paling banyak digunakan adalah python2 dan 3, di setiap versi memiliki pengembangan tertentu, perbedaan python 2 dan 3 yaitu ketika membuat codingan.
	
	\begin{enumerate}
\item Syntax untuk mencetak sebuah teks Di python 2 kita bisa memakai tanda kurung atau tidak, namun di python 3 harus menggunakan tanda kurung jika tidak akan terjadi error.
\item Syntax untuk meminta inputan Pada python 2 kita harus menggunakan “raw\_input(“teks”)”, pada python 3 kita hanya perlu menulis syntax”input(“teks”)”.
	
	\end{enumerate}
\par	Karena syntax di python 2 sudah tidak compatibel di shell python 3. 
