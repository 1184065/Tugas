\chapter*{Perusahaan Dunia yang menggunakan bahasa pemrograman Python}

\section*{\textit{Spotify}}
\par
\textit{Spotify} adalah suatu layanan musik streaming yang memanfaatkan bahasa pemprograman python untuk analisis data dan backend. Pada backend spotify berkomunikasi dengan 0MQ. 0MQ itu sendiri adalah suatu framework dan library open source untuk networking. Untuk menginterpretasikan analisis data tersebut spotify menggunakan luigi, dan modul python yang sinkron dengan hadoop. Modul open source ini menangani satu library dengan library lainnya agar saling bekerjasama, serta dapat mengkonsolidasi eror log secara cepat.
\section*{\textit{Google}}
\par
\textit{Google} ini sudah menggunakan bahasa pemprograman python ini sudah sajak dari awal berdirinya. Dan pada saat ini bahasa pemprograman python merupakan salah satu bahasa pemprograman server-side resmi di google. Meskipun ada script yang ditulis untuk google menggunakan bahasa perl dan bash, maka nantinya script tersebut akan diubah ke python terlebih dahulu, karena kemudahan dalam perawatannya.

\section*{\textit{Industrial Light and Magic}}
\par
\textit{Industrial Light and Magic} merupakan studio special efek yang dibutuhkan untuk film star wars saja. Karena infrastruktur awal industrial light and magisc ini menggunakan C dan C++, maka akan lebih mudah mengintegrasikan bahasa pemprograman python ketimbang bahasa pemprograman lainnya. Dengan menggunakan bahasa pemprogramana python ini industrial light and magic dengan mudah membungkus komponen software dan dapat meningkatkan aplikasi grafisnya.

\section*{\textit{Netflix}}
\par
\textit{Netflix} adalah suatu layanan pemutaran film yang dapat dilakukan oleh pengguna dimanapun dan kapanpun. Pada netfilx bahasa pemprograman yang digunakan adalah bahasa pemprograman python, bahasa pemprograman ini digunakan pada Central Alert Gateway yang akan me-reroute alert dan mengirimkannya pada individu yang akan melihatnya serta juga  dapat secara otomatis reboot atau menghentikan proses yang dianggap bermasalah. Selain itu python juga digunakan untuk menelusuri riwayat dan perubahan pengaturan keamanan.

\section*{\textit{Instagram}}
\textit{Instagram} adalah suatu aplikasi mobile berbasis IOS, android dan windows phone, dimana pengguna dapat berbagi foto dan video melalui instagram ini. Pada instagram ini menggunakan bahasa pemprograman python dalam task queuennya atau fitur dimana setiap pengguna dapat berbagi foto atau video ke beberapa social network lainnya seperti facebook, twitter, dan lain-lainnya.

Selain perusahaan diatas ada beberapa perusahaan pengguna Python lain yaitu: Pinterest, Disqus, Dropbox, Uber, Reddit, Quora, Facebook (Bahasa ke-3 setelah PHP (Hack) dan C++, digunakan untuk manajemen infrastruktur).
\par

