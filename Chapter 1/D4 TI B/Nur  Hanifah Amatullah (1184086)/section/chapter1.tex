\chapter*{SEJARAH PYTHON}

\par
	Python adalah bahasa pemprograman yang dinamis, yang mendukung pemprograman berbasis objek. Python itu sendiri dikembangankan oleh Guido Van Rossum pada tahun 1990-an di CWI, Amsterdam. Bahasa pemprograman ini merupakan kelanjutan dari bahasa pemprograman ABC. Nama python itu sendiri diambil dari kegemaran Guido Van Rossum pada salah satu acara humor ditelevisi pada era 1980-an yang berjudul “Monty Python’s Flying Circus”. Bahasa pemprograman python ini menggunakan metode pemprosesan interpreted, yaitu kode pemprograman akan diproses baris per baris langsung dari kode program. Dan bahasa pemprograman ini juga disebut sebagai bahasa pemprograman tingkat tinggi serta bisa dipakai untuk berbagai jenis tujuan. Tahun 1995, Guido pindah ke CNRI di Virginia Amerika untuk melanjutkan perkembangan python itu sendiri, dan merilis beberapa versi terbaru dari python. Hampir semua penggembangan python dirilis menggunakan lisensi GFL-compatible.

\par
	Python yang banyak digunakan sekarang adalah Python 2 dan Python 3, Namun disetiap pengembangan versi tentu saja memiliki peningkatan kualitas seperti yang terjadi di Pyhton 2 dan Python 3. Perbedaan python 2 dan python 3 yaitu ketika membuat kodingan di Phyton 2 dan di complide di shell python 3 akan terjadi eror karena script python 2 sudah tidak compatibel di shell python 3, karena di python 3 memerlukan tanda kurung () sedangkan di python 2 tidak memerlukannya. Di python 3 akan terlihat lebih rapih dibandingan di python 2. . Dan pada python 2 dilengkapi dengan berbagai fitur programatikal sedangkan pada python 3 itu sendiri melakukan perapian pada codebase dan penghapusan redundancy.