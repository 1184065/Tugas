\documentclass[12pt]{ociamthesis}  % default square logo 
%\documentclass[12pt,beltcrest]{ociamthesis} % use old belt crest logo
%\documentclass[12pt,shieldcrest]{ociamthesis} % use older shield crest logo

%load any additional packages
\usepackage{amssymb}
\usepackage{multirow}
\usepackage{caption}
\usepackage{titlesec}
\usepackage{charter}
\usepackage{subfiles}
\usepackage{etoolbox} %halaman I-1
%\patchcmd{\addcontentsline}
 % {\thepage}
  %{\thechapter-\thepage}
  %{}
  %{}

\makeatletter
\@addtoreset{subsubsection}{chapter}
\makeatother
\renewcommand{\thepage}{\thechapter-\number\numexpr\arabic{page}\relax}
\patchcmd{\cleardoublepage}{\ifodd}{\unless\ifodd}{}{}
\preto\chapter{\cleardoublepage}
\AtBeginDocument{\setcounter{page}{1}}





\makeatletter					%
\renewcommand*\@pnumwidth{3em}	% ratain halaman daftar isi
\makeatother					%

\title{\large RESUME PYTHON\\[1ex]     %your thesis title,
        }   %note \\[1ex] is a line break in the title
        

\author{Diajukan untuk memenuhi kelulusan matakuliah Pemrograman II 
pada Program Studi DIV Teknik Informatika\\[5ex]
O l e h : \\
[1ex] Nur Hanifah Amatullah   \\ 
\textbf{1.18.4.086}\\
		 [5ex] } 
		 
\college{}  %your college

%\renewcommand{\submitt edtext}{change the default text here if needed}
\degree{POLITEKNIK POS INDONESIA}    %the degree
\degreedate{\textbf{BANDUNG\\2019}}  

%end the preamble and start the document
\begin{document}

%this baselineskip gives sufficient line spacing for an examiner to easily
%markup the thesis with comments
\baselineskip=18pt plus1pt

%set the number of sectioning levels that get number and appear in the contents
\setcounter{secnumdepth}{3}
\setcounter{tocdepth}{3}


\maketitle          % create a title page from the preamble info

   
\chapter{Mengenal Python dan Anaconda}
Tujuan pembelajaran pada pertemuan pertama antara lain:
\begin{enumerate}
\item
Mengerti sejarah python, perkembangan dan penggunaan python di perusahaan
\item
Memahami tahapan instalasi python dan anaconda
\item
Memahami cara penggunaan spyder
\end{enumerate}
Tugas dengan cara dikumpulkan dengan pull request ke github dengan menggunakan format latex pada repo yang dibuat oleh asisten IRC.

\section{Teori}
Praktek teori penunjang yang dikerjakan :
\begin{enumerate}
\item
Buat Resume Sejarah Python, perbedaan python 2 dan 3, dengan bahasa yang mudah dipahami dan dimengerti. Buatan sendiri bebas plagiat(10)
\item
Buat Resume Implementasi dan penggunaan Python di perusahaan dunia, bahasa yang mudah dipahami(10)
\end{enumerate}

\section{Instalasi}
Melakukan instalasi python dan anaconda versi 3 serta uji coba spyder. Dengan menggunakan bahasa yang mudah dimengerti dan bebas plagiat. 
Dan wajib skrinsut dari komputer sendiri.
\begin{enumerate}
\item
Instalasi python 3 (5)
\item
instalasi pip(5)
\item
cara setting environment (5)
\item
mencoba entrepreter/cli melakui terminal atau cmd windows(5)
\item 
Menjalankan dan mengupdate anaconda dan spyder(5)
\item
Cara menjalankan Script hello word di spyder(5)
\item
Cara menjalankan Script otomatis login aplikasi akademik dengan library selenium dan inputan user(5)
\item
Cara pemakaian variable explorer di spyder(5)
\end{enumerate}


\section{Identasi}
Membuat file main.py dan mengisinya dengan script contoh python penggunaan selenium(minimal 20 baris) yang melibatkan inputan user, kemudian mencoba untuk mengatasi error identasi.
\begin{enumerate}
	\item
Penjelasan Identasi (10)
	\item
jenis jenis error identasi yang didapat(10)
\item
cara membaca error(10)
\item 
cara menangani errornya(10)
\end{enumerate}

\section{Presentasi Tugas}
Pada pertemuan ini, diadakan tiga penilaiain yaitu penilaian untuk tugas mingguan dengan nilai maksimal 100. Kemudian dalam satu minggu kedepan maksimal sebelum waktu mata kuliah. Ada presentasi kematerian dengan nilai presentasi yang terpisah masing-masing 100. Dan nilai terpisah untuk tutorial dari jawaban tugas di YouTube.Jadi ada tiga komponen penilaiain pada pertemuan ini yaitu :
\begin{enumerate}
	\item tugas minggu hari ini dan besok (maks 100). pada chapter ini
	\item presentasi csv (maks 100). Mempraktekkan kode python dan menjelaskan cara kerjanya.
	\item pembuatan video tutorial youtube tentang tutorial dari jawaban tugas.(nilai maks 100)
\end{enumerate}
Waktu presentasi pada jam kerja di IRC. Kriteria penilaian presentasi sangat sederhana, presenter akan ditanyai 20(10 pertanyaan program, 10 pertanyaan teori) pertanyaan tentang pemahamannya menggunakan python dan program agan dibuat error hingga presenter bisa menyelesaikan errornya. jika presenter tidak bisa menjawab satu pertanyaan asisten maka nilai nol. Jika semua pertanyaan bisa dijawab maka nilai 100. Presentasi bisa diulang apabila gagal, sampai bisa mendapatkan nilai 100 dalam waktu satu minggu kedepan.

\chapter*{Perusahaan Dunia yang menggunakan bahasa pemrograman Python}

\section*{\textit{Spotify}}
\par
\textit{Spotify} merupakan layanan musik streaming yang sudah banyak digunakan di seluruh dunia. Dalam bidang menganalisis data spotify menggunakan Bahasa Pemrograman Python, dalam pengimplementasiannya Tim Spotify menggunakan Luigi, modul yang ada di Python yang disingkronisasikan pada sebuah software yang memudahkan programmer membuat aplikas web atau disebut framework yang berbasis Java yang memungkinkan pemrosesan data dalam waktu cepat.Penerapan bahasa Python juga digunakan dalam penerapan fitur Radio dan Discover serta fitur merekomendasikan orang yang mungkin akan diikuti.

\section*{\textit{Netflix}}
\par
\textit{Netflix} merupakan layanan pemutaran film atau tayangan yang memungkinkan para penggunaknya menggunakan di manapun dan kapanpun. Netflix menggunakan bahasa pemrograman Python. Penggunaan Python di Netflix terdapat pada Central Alert Gateaway (C.A.G) ini akan me-reroute alert dan mengirimkannya pada kelompok atau individu yang dapat melihatnya dan secara otomatis reboot atau menghentikan proses yang dianggap bermasalah dan digunakan untuk menulusuri riwayat dan perubahan pengaturan keamanan. Tetapi sama halnya seperti spotify, Netflix menggunakan python untuk menganilisis data dan lebih utama terlihat pada bagian bagaimana netflix merekomendasikan film kepada pelangganya.

\section*{\textit{Pinterest}}
\par
\textit{Pinterest} adalah aplikasi web yang digunakan untuk mengumpulkan hal-hal yang menarik berdasarkan kriteria tertentu yang sering dikunjungi di jaman sekarang. Pinterest menggunakan Bahasa Pemrograman Python dari awal mereka membangunnya itulah sebabnya bookmarking (Sebuah metode bagi pengguna internet untuk mengorganisasi, menyimpan, mengelola, dan mecari penanda sumber daya yang tersedia secara online) yang ada di pinterest begitu terstruktur dan mudah untuk diatur.

\section*{\textit{Instagram}}
\par
\textit{Instagram} merupakan sebuah aplikasi berbagi foto dan video secara digital yang digunakan oleh lebih dari 400 juta user yang aktif setiap harinya. Instagram menggunakan bahasa pemrograman python dalam task queuenya atau fitur dimana pada saat yang bersamaan instagram dapat melakukan posting ke beberapa social network lainnya seperti Facebook, Twitter, dll. 

\section*{\textit{Industrial Light and Magic}}
\textit{Indusrial Light and Magic} adalah Studio spesial-efek yang digunakan pada pemutaran efek di film Star Wars. Dalam pembuatan efek ledakan ILM menggunakan Python dikarenakan dapat menghemat waktu dalam pembuatan efek tersebut.   
\par







\end{document}

