

\chapter*{Chapter 1}
\section*{Sejarah Python}
\par

 
Bahasa pemrograman Python merupakan bahasa pemrograman yang dibuat Guido van Rossum. Pembuatan bahasa Python untuk membuat skrip bahasa tingkat tinggi pada sebuah sistem operasi yang terdistribusi Amoeba. Python sudah digunakan oleh banyak pengembang dan juga sudah digunakan oleh banyak perusahaan untuk pembuatan perangkat lunak komersial.
Bahasa Python merupakan pemrogram gratis atau freeware, dapat dikembangkan, dan tidak ada batasan dalam penyalinannya dan mendistribusikan. Beberapa pelayanan yang terdapat didalamnya yaitu disediakan lengkap dengan source codenya, debugger dan profiler, interface, fungsi sistem, GUI, dan basisdatanya. Python tersedia untuk berbagai Sistem Operasi, seperti Unix (linux), PCs (DOS, Windows, OS/2), Machintosh dan masih banyak lainnya.

\par
    
 Seperti kita ketahui bahasa Python terdapat dari 2 jenis yaitu Python 2 dan Python 3. Pada kedua versi ini memiliki perbedaan pada fungsi print Code. Kita tidak memerlukan tanda kurung dalam proses print code pada Python2. Pada Python 2 kita hanya menulis print “hello world”. Tetapi pada Python 3 kita harus menulis print(”hello world”)

