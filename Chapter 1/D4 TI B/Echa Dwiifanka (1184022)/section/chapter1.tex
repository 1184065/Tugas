\chapter*{ SEJARAH PYTHON}

\par
	Python  pertama kali dikembangkan oleh Guido van Rossum pada tahun 1990 di CWI, Amsterdam  yang dijadikan sebagai kelanjutan dari bahasa pemrograman ABC. Versi terakhir yang dikeluarkan CWI adalah 1.2. saat ini pengembangan python terus dilakukan oleh beberapa pemrogram yang dikoordinasikan oleh Giido dan  Python Software Foundation. Python Software Foundation adalah sebuah organisasi non-profit yang dibangun sebagai hak cipta intelektual Python sejak versi 2.1 dan dapat mencegah Python akan dimiliki oleh beberapa perusahaan komersial. Seiring perkembangan zaman pada saat ini distribusi Python sudah mencapai versi 2.6.1 dan versi 3.0.
\par
    Semua versi python yang dirilis semuanya bersifat open source. Dalam sejarahnya hamper semua python menggunakan lisensi GFL-compatible. Berikut ini akan dijelaskan beberapa versi python  yang lengkap dengan tanggal rilisnya.

		
\begin{enumerate}
\item Pyton 1.0 – 1.6 Rilis dari tahun 1994- pertengahan 2000
\item Pyton 2.0 – 2.7 Rilis dari tahun pertengahan 2000-2010 
\item Pyton 3.0 – 3.7 Rilis dari tahun 2008-2018
\end{enumerate}

\par
	2.	Perbedaan python 2 dan python 3
Perbedaan dari python 2 dan python 3 dapat dibagi menjadi beberapa perbedaan antara lain sebagai berikut:


1.	Pada python versi 2 dalam pembuatan syntaxnya  dapat menentukan apakah ingin menggunakan tanda kurung atau tidak, sedangkan pada python 3 diwajibkan untuk menggunakan tanda kurung.
2.	ka kita ingin mementa sebuah inputan pada python 2, kita harus menggunakan syntax “rawinput(‘teks’)”. Dan untuk python versi 3, kita cuma perlu syntax “input(‘teks’)”.


\chapter*{ PENGIMPLEMENTASIAN PYTHON PADA PERUSAHAAN DUNIA}

\par
Pengimplementasian bahasa pemrograman python pada perusahaan dunia terdapat  beberapa  perusahaan yang menggunakannya antara lain :

1.	Google adalah perusahaan besar yang menggunakan banyak kode Python     di dalam mesin pencarinya. Dan 			mesin pencari google adalah     yang paling terkenal di dunia.

2.	Youtube, situs video terbesar dan terpopuler di dunia, sebagian         besar kodenya ditulis dalam bahasa Python.

3.	Facebook, media sosial terbesar di dunia, menggunakan Tornado, sebuah framework Python untuk 				menampilkan timeline.

4.	Instagram, siapa yang tidak kenal. Instagram menggunakan Django,        framework python sebagai mesin 			pengolah sisi server dari       aplikasinya.
5.	Pinterest, banyak menggunakan python untuk membangun aplikasinya.

6.	Dropbox, barangkali Anda adalah salah seorang pengguna layanan ini.     Dropbox menggunakan python baik di 	sisi server maupun di sisi          pengguna layanannya.

7.	Quora, salah satu situs tanya jawab terbesar di dunia, dibangun         menggunakan Python.
8.	NASA, badan antariksa Amerika ini menggunakan Python untuk bidang       sainsnya.
