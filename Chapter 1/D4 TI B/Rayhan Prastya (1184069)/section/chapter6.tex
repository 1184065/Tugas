\chapter{KRITERIA HASIL SIDANG}

\section{Lulus}
Mahasiswa	dinyatakan	lulus	sidang Proyek	jika	:
\begin{enumerate}
	\item Menjalani	sidang dengan	lancar.
	\item Tidak	terjadi	revisi,	baik aplikasi	maupun	laporan	proyek.
	\item Mendapat	nilai	dengan	rata-rata	dari	penguji	minimum	60.	
\end{enumerate}

\section{Lulus Bersyarat}
Mahasiswa	dinyatakan	lulus	bersyarat	dari	sidang Proyek	jika	:
\begin{enumerate}
\item Memenuhi	persyaratan	lulus.
\item Ada	 tugas	 yang	 diberikan	 penguji,	 jika	 tugas	 tersebut	 	 telah	 selesai	 dikerjakan	 maka	
statusnya	otomatis	berubah	menjadi	Lulus.	\\
Tugas	tersebut	berupa	:
\begin{itemize}
\item Perbaikan Laporan.
\item Perbaikan Aplikasi.
\end{itemize}
\end{enumerate}

\section{Tidak Lulus}
Mahasiswa	dinyakan	tidak	lulus	sidang Proyek	jika	:
\begin{itemize}
\item Ditemukan	hasil \textit{plagiat} 80\%	dari	Proyek	I,	II	yang	pernah	disidangkan.
\item Tidak	memenuhi	kriteria	Lulus.
\item Tidak	datang	pada	waktu	sidang	 tanpa	informasi	apapun,	sedangkan	 tim	penguji	sudah	siap	di	lokasi	sidang.
\end{itemize}

\section{Penilaian}
\subsection{Komponen	Nilai}
Nilai	Proyek	tersedia	atas	dua	komponen	nilai yaitu	:
\begin{itemize}
	\item Nilai	buku	dan	bimbingan.
	\item Nilai	sidang Proyek.
\end{itemize}

\subsection{Nilai	Buku	dan	Bimbingan}
Nilai	buku	dan	bimbingan	dikeluarkan	oleh	pembimbing,	dengan	skala	nilai	1-100. Nilai	
akhir	buku	dan	bimbingan	adalah	nilai	rata-rata	dari	seluruh	pembimbing.Paling	lambat	2	
hari	 sebelum	 sidang,	 mahasiswa	 wajib	 menyerahkan	 nilai	 dari	 pembimbing	 ke	
koordinator	proyek.

\subsection{Nilai	Sidang Proyek	Pemrograman	dan	Jaringan (PROYEK	II).}
Nilai	 ini	 dikeluarkan	 oleh	 penguji,	 dengan	 skala	 nilai	 1-100	 segera	 setelah	 sidang
dilaksanakan,	 ketua	 sidang wajib	 menyerahkan	 hasil	 penilaiannya.Nilai	 Proyek	 adalah	
35\%	dari	rata-rata	nilai	pembimbing	dan	65\%	dari	rata-rata	nilai	Penguji.

\subsection{Pengolahan	Nilai}
Pengolahan	 nilai	 dilaksanakan	 oleh	 Koordinator	 Proyek.	 Dalam	 menjalankan	 tugas	pengolahan	nilai,	Koordinator	Proyek	mempunyai	hak	penuh	 yang	 tidak	dapat	diganggu	
gugat	 oleh	 siapa	 pun.	 Pengolahan	 nilai	 dilaksanakan	 berdasarkan	 nilai	 yang	 diberikan	
oleh	pembimbing	dan	penguji	sidang Proyek.	Adapun	rumus	yang	dipakai	adalah	sebagai	
berikut	: \\

\textbf{Nilai	Akhir	=			(35\%	x	rata-rata	nilai	bimbingan)	+	(65\% x	rata-rata	nilai	sidang)	} \\

\textbf{Index Nilai :} \\
$85	\leq Nilai	\leq 100$	= A \\
$71	\leq Nilai	\leq 84$	= B \\
$56	\leq Nilai	\leq 70	$	= C \\

\subsection{Distribusi	Hasil	Pengolahan	Nilai}
Nilai	 Proyek	 Program	 disampaikan	 ke	 mahasiswa	 yang	 bersangkutan	 dan	 Ketua	 Prodi oleh	 Koordinator	 Proyek.	 Nilai	 tersebut	 akan	 keluar,	 jika	 persyaratan	 keluarnya	 nilai	Proyek	 telah	 terpenuhi.	 Koordinator	 Proyek	 mempunyai	 hak	 penuh	 untuk	 tidak	mengeluarkan	nilai	Proyek	jika	mahasiswa	 tidak	memenuhi	persyaratan	keluarnya	nilai	Proyek.	Adapun	nilai	Proyek	akan	dikeluarkan	oleh	Koordinator	Proyek	jika	:
\begin{enumerate}
	\item Buku	telah	dijilid	dan	didistribusikan	sesuai	dengan	ketentuan.
	\item . Tidak		ada			permasalahan		dengan		pinjaman		fasilitas	dan	alat	yang	digunakan	selama	
pelaksanaan	Proyek .
	\item Menyerahkan	 alat	 hasil	 Proyek	 bagi	 mahasiswa	 yang	 telah	 menyatakan	 kesediaanya	untuk	menyumbangkan	alat	tersebut .
	\item Mengumpulkan	CD	Proyek .
	\item Mengumpulkan	Jurnal Proyek	dalam	bahasa	Indonesia	dan	Inggris.	
\end{enumerate}