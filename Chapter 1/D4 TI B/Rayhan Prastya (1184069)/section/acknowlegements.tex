\begin{acknowledgements}
Pertama-tama	 kami	 panjatkan	 puji	 dan	 syukur	 kepada	Allah	SWT	 yang	 telah	memberikan	rahmat	 dan	 hidayah-Nya	 sehingga	 Buku	 Pedoman	 dan	 Kegiatan	 Proyek	Debugger Tangguh (PROYEK	II)	ini	dapat	diselesaikan.

Buku	 Pedoman	 ini	 dibuat	 dengan	 tujuan	 memberikan	 acuan,	 baik	 bagi	 mahasiswa	 yang	
akan	mengambil	matakuliah	Proyek	Debugger Tangguh	(PROYEK	II)	maupun	bagi	dosen	pembimbing.	Pada	intinya	buku	ini	menjelaskan	secara	lengkap	tentang	Karakteristik	PROYEK	II	
di	 Program	 Studi	 Teknik	 Informatika,	 dan	 juga	 mengatur	 mekanisme,	 teknik	 penulisan,	 serta	
penilaiannya.Dengan	demikian	diharapkan	semua	pihak	yang	terlibat	dalam	aktivitas	PROYEK	II	
mempunyai	kesamaan	dalam	pelaksanaannya

Tak	 ada	 gading	 yang	 tak	 retak,	 tak	 ada	 manusia	 yang	 sempurna	 dan	 apapun	 yang	
dihasilkannya,	sehingga	koreksi	serta	masukan	untuk	berbagai	kekurangan	dalam	Buku	Pedoman	
dan	 Kegiatan	 Pemrograman	 dan	 Jaringan	 (PROYEK	 II)	 ini	 tetap	 diharapkan.	 Terimakasih	 atas	
kerjasama	 banyak	 pihak,	 dan	 semoga	 buku	 ini	 memberikan	 banyak	 manfaat	 khususnya	 bagi	
pihak-pihak	yang	terkait.


\begin{raggedright}

Bandung,  September 2019 \\
Ketua	Prodi	DIV	Teknik	Informatika\\[8ex]


\underline{M.	Yusril	Helmi	Setyawan,	S.Kom.,	M.Kom.}\\
NIK.113.74.163

\end{raggedright}

\end{acknowledgements}