\documentclass{article}
\usepackage[utf8]{inputenc}

\title{Sejarah Python, Perbedaan Python 2 dan Python 3, Implementasi Bahasa Python dan Kegunaan Python di Perusahaan Dunia }

\author{Mauliddhia Restu Shafina (1184101) }
\date{18 Oktober 2019}

\begin{document}

\maketitle

\section{Sejarah Python}

\usepackage{Diciptakan oleh Guido van Rossum pada awal tahun 1990 di  Centrum Wiskunde & Informatica (CWI) Belanda. Berawal dari bahasa pemrograman ABC, yang sekarang bersifat open source sehingga ribuan orang juga bisa berkontribusi dalam perkembangannya. Dia merilis beberapa versi dari python pada tahun 1950 di Corporation for National Research Initiative (CNRI), Virginia Amerika. Guido van Rossum dan tim inti python berpindah ke BeOpen.com membentuk tim beOpen.com pythonLabs dan mengeluarkan python 2.0. Ditahun yang sama Guido van Rossum beserta tim inti python berpindah lagi ke Digital Creation.} 

\usepackage{Pada saat ini pengembangan Python terus dilakukan oleh sekumpulan pemrogram yang dikoordinir Guido dan Python Software Foundation. Organisasi Python Software Foundation adalah sebuah organisasi non-profit yang dibentuk sebagai pemegang hak cipta intelektual Python sejak versi 2.1 dan dengan demikian mencegah Python dimiliki oleh perusahaan komersial. Karena kecintaan guido pada acara televisi Monty Python's Flying Circus, Guido van Rossum memilih nama Python untuk dijadikan nama bahsa pemrograman yang dibuatnya.}

\section{Perbedaan Python 2 Dan Python 3}

\usepackage{Ada beberapa perbedaan, antara lain}

\usepackage{-	Ketika akan di print dengan menampilkan hello world. Perbedaan ini membuat sintaksis pada Python lebih konsisten. Dengan menggunakan  print()  juga kompatibel dengan Python 2.7. }

\usepackage{Contoh =}

\usepackage{Source code python 2 : print "Hello world, python"}

\usepackage{
Source code python 3 : print ("Hello world, python") }

\usepackage{
-	Pembagian pada integer. Python 2, semua tipe data angka tidak mengandung desimal akan di perlakukan sebagai integer. Dan jika kedua integer tersebut dibagi akan menghasilkan data float.}

\usepackage{Contoh =}

\usepackage{ 3/2 = 1,5}
 
\usepackage{
-	Pada python 2, dibulatkan ke nilai paling rendah misalnya 1.5 jadi 1, 2.6 jadi 2 dan seterusnya atau disebut juga floor devision.}

\usepackage{ Contoh = }

\usepackage{x= 3/2}

\usepackage{print a}

\usepackage{#Output}

\usepackage{1}

\usepackage{
 -	Pada python 3 untuk mendapatkan nilai desimal seperti 1.5, tambahkan menjadi 3.0 / 2.0 }
 
 \usepackage{contoh =}
 
 \usepackage{a= 3/2}
 
 \usepackage{print(a)}
 
 \usepackage{#Output}
 
 \usepackage{1,5}

\usepackage{ 
-	Pada python 3 juga, untuk mendapatkan floor devision gunakan “//”}

\usepackage{ Contoh =}

\usepackage{b= 3//2}

\usepackage{print (b)}

\usepackage{#Output}

\usepackage{1}
 
\section{Implementasi Bahasa Python}

\usepackage{Implementasi python asli adalah Cpython. Mengapa disebut dengan Cpython? untuk membedakannya dari implementasi Python yang lain. Lalu, untuk membedakan penerapan bahasa mesin dari bahasa pemrograman Python itu sendiri. Kebetulan, CPython di implementasikan di C yang mengkompilasi kode python kamu menjadi bytecode (transparan) dan menafsirkan bytecode dalam lingkaran evaluasi.}

\usepackage{Bukan hanya CPython. Jython, IronPython dan PyPy adalah implementasi 'lain' dari bahasa pemrograman Python. Kode python bisa berjalan di JVM karena Jython mengkompilasi kode Python ke Java bytecode. Jika IronPython memungkinkan kamu menjalankan Python di Microsoft CLR. Diimplementasikan di (subset dari) Python, yang memungkinkan kamu menjalankan kode Python lebih cepat dari CPython menggunakan JIT Compiler adalah PyPy.}

\section{Kegunaan python di perusahaan dunia}

\usepackage{Bahasa python sering di pakai oleh data analis, data scientists dan para Software Engineers untuk menjalankan proses pembangunan sebuah aplikasi dan untuk menggali lebih dalam machine learning. Banyak perusahaan yang dibuat menggunakan bahasa pemrograman python diantaranya google, spaceX dan Instagram. Ada juga beberapa hal yang sudah dibangun menggunakan python, yaitu Artificial Intellegence(AI) , aplikasi web contohnya pinterest, special effect pada Hollywood. }
\end{document}
