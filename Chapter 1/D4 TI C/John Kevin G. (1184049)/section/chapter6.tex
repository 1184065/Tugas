\chapter{Matplotlib}

Tujuan pembelajaran pada pertemuan kelima antara lain:
\begin{enumerate}
\item
Mengenal plot di python
\item
Mengerti cara memakai library Matplotlib
\item
Mengerti cara memplot berbagai macam jenis plot
\item
Mengatasi Error yang terjadi akibat pemakaian matplotlib
\item
Try Except
\end{enumerate}
Tugas dengan cara dikumpulkan dengan pull request ke github dengan menggunakan latex pada repo yang dibuat oleh asisten IRC. Kode program dipisah dalam folder src NPM.py yang berisi praktek dari masing-masing tugas file terpisah sesuai nomor yang kemudian dipanggil menggunakan input listing ke dalam file latex penjelasan atau nomor pengerjaan. Masing masing soal bernilai 5 dengan total nilai 100. Gunakan bahasa yang baku dan bebas plagiat dengan dibuktikan hasil scan plagiarisme. Serta hasil scrinsut dari komputer sendiri, dan kode hasil sendiri. Pengerjaan menggunakan latex dan harus menyertakan file pdf hasil compile pdflatex, jika tidak diskon 50\%.


\section{Pemahaman Teori}
Kerjakan soal berikut ini, masing masing bernilai 5. Untuk hari pertama.
Praktek teori penunjang yang dikerjakan dengan deadline hari pertama jam 4 pagi:
\begin{enumerate}
\item
Apa itu fungsi library matplotlib
\item
Jelaskan langkah-langkah membuat sumbu X dan Y di matplotlib
\item
Jelaskan bagaimana perbedaan fungsi dan cara pakai untuk berbagai jenis(bar,histogram,scatter,line dll) jenis plot di matplotlib
\item
Jelaskan bagaimana cara menggunakan legend dan label serta kaitannya dengan fungsi tersebut
\item
Jelaskan apa fungsi dari subplot di matplotlib, dan bagaimana cara kerja dari fungsi subplot, sertakan ilustrasi dan gambar sendiri dan apa parameternya jika ingin menggambar plot dengan 9 subplot di dalamnya
\item
Sebutkan semua parameter color yang bisa digunakan (contoh: m,c,r,k,... dkk)
\item
Jelaskan bagaimana cara kerja dari fungsi hist, sertakan ilustrasi dan gambar sendiri
\item
Jelaskan lebih mendalam tentang parameter dari fungsi pie diantaranya labels, colors, startangle, shadow, explode, autopct
\end{enumerate}

\section{Ketrampilan Pemrograman}
Kerjakan soal berikut ini, masing masing bernilai 10 untuk hari kedua jam 4 pagi. Soalnya adalah:

\begin{enumerate}
\item
Buatlah librari fungsi (file terpisah/library dengan nama NPM\_bar.py) untuk plot dengan jumlah subplot adalah NPM mod 3 + 2
\item
Buatlah librari fungsi (file terpisah/library dengan nama NPM\_scatter.py) untuk plot dengan jumlah subplot NPM mod 3 + 2
\item
Buatlah librari fungsi (file terpisah/library dengan nama NPM\_pie.py) untuk plot dengan jumlah subplot NPM mod 3 + 2
\item
Buatlah librari fungsi (file terpisah/library dengan nama NPM\_plot.py) untuk plot dengan jumlah subplot NPM mod 3 + 2
\end{enumerate}




\section{Ketrampilan Penanganan Error}
Kerjakan soal berikut ini, masing masing bernilai 5(hari kedua). Bagian Penanganan error dari script python.
\begin{enumerate}
\item
Tuliskan peringatan error yang didapat dari mengerjakan praktek ketiga ini, dan jelaskan cara penanganan error tersebut.
dan Buatlah satu fungsi yang menggunakan gunakan try except untuk menanggulangi error tersebut.
\end{enumerate}



\section{Presentasi Tugas}
Pada pertemuan ini, diadakan dua penilaiain yaitu penilaian untuk tugas mingguan seperti sebelumnya dengan nilai maksimal 100. Kemudian dalam satu minggu kedepan maksimal sebelum waktu mata kuliah pemrograman 3. Ada presentasi kematerian dengan nilai presentasi yang terpisah masing-masing 100. Jadi ada tiga komponen penilaiain pada pertemuan ini yaitu :
\begin{enumerate}
	\item tugas minggu hari ini dan besok (maks 100). pada chapter ini
	\item presentasi matplotlib (maks 100). Mempraktekkan kode python dan menjelaskan cara kerjanya.
\end{enumerate}
Waktu presentasi pada jam kerja di IRC. Kriteria penilaian presentasi sangat sederhana, presenter akan ditanyai 20(10 pertanyaan program, 10 pertanyaan teori) pertanyaan tentang pemahamannya menggunakan python untuk kecerdasan buatan. jika presenter tidak bisa menjawab satu pertanyaan asisten maka nilai nol. Jika semua pertanyaan bisa dijawab maka nilai 100. Presentasi bisa diulang apabila gagal, sampai bisa mendapatkan nilai 100 dalam waktu satu minggu kedepan.




