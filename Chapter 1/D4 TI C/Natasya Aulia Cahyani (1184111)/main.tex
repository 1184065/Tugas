\documentclass{article}
\usepackage[utf8]{inputenc}

\title{Tugas Resume Pemograman II}
\author{ Oleh Natasya Aulia Cahyani }
\date{16 October 2019}

\begin{document}

\maketitle

\section{Sejarah Python}

\usepackage{Python dikembangkan oleh Guido van Rossum pada tahun 1990 di CWI, Amsterdam sebagai kelanjutan dari bahasa pemrograman ABC. Versi terakhir yang dikeluarkan CWI adalah 1.2.}

\usepackage{Tahun 1995, Guido pindah ke CNRI sambil terus melanjutkan pengembangan Python. Versi terakhir yang dikeluarkan adalah 1.6. Tahun 2000, Guido dan para pengembang inti Python pindah ke BeOpen.com yang merupakan sebuah perusahaan komersial dan membentuk BeOpen PythonLabs. Python 2.0 dikeluarkan oleh BeOpen. Setelah mengeluarkan Python 2.0, Guido dan beberapa anggota tim PythonLabs pindah ke DigitalCreations.}

\usepackage{Saat ini pengembangan Python terus dilakukan oleh sekumpulan pemrogram yang dikoordinir Guido dan Python Software Foundation. Python Software Foundation adalah sebuah organisasi non-profit yang dibentuk sebagai pemegang hak cipta intelektual Python sejak versi 2.1 dan dengan demikian mencegah Python dimiliki oleh perusahaan komersial. Saat ini distribusi Python sudah mencapai versi 2.6.1 dan versi 3.0.
Nama Python dipilih oleh Guido sebagai nama bahasa ciptaannya karena kecintaan guido pada acara televisi Monty Python s Flying Circus. Oleh karena itu seringkali ungkapan-ungkapan khas dari acara tersebut seringkali muncul dalam korespondensi antar pengguna Python.}

\section{Apa itu Python?}
\usepackage{Python merupakan bahasa pemrograman yang populer dan banyak digunakan oleh Data Analysts, Data Scientists dan para Software Engineers untuk menjalankan proses pembangunan sebuah aplikasi dan untuk menggali lebih dalam machine learning. Perusahaan seperti google, spaceX dan Instagram membutuhkannya untuk membersihkan data, membangun prediksi, model untuk AI, web app, dan masih banyak lagi. Berikut adalah beberapa hal yang dibangun dari Python dan sudah kamu rasakan dalam kehidupan sehari-hari:

\usepackage{Artificial Intelligence (AI)
atau yang biasa disebut dengan kecerdasan buatan ini merupakan tombak utama dari AI. Contohnya adalah saat pebisnis membutuhkan chat otomatis untuk menjawab pertanyaan dari para costumer-nya. Untuk hal ini programmer dapat menggunakan Python untuk mengidentifikasi email yang masuk beserta jawabannya dan kemudian memvisualkan jawaban yang diinginkan oleh pebisnis dan merancang prediksi jawaban yang diinginkan. Python juga dapat digunakan untuk memprediksi model jawaban melalui aplikasi chatbot untuk respons yang lebih cepat tanggap.}

\usepackage{Aplikasi Web
Python merupakan bagian penting dari beberapa website yang paling banyak dikunjungi abad ini seperti Pinterest, Instagram, Spotify, dan Youtube. Faktanya Pinterest sudah menggunakan Python dari awal web ini dibangun. Itulah mengapa bookmarking oleh halaman web ini begitu terstruktur dan mudah untuk diatur.}

\usepackage{Special Effect pada Hollywood
Ingatkah kamu adegan ledakan dalam film action? Ada banyak perusahaan yang bergerak di bidang perfilman seperti Lucasfilm yang menggunakan Python sebagai sebagai special effect. Dengan menggunakan Python perusahaan perfilman dapat menghemat banyak waktu dalam membuat effect.}

\usepackage{Menggunakan Python dalam memprediksikan kemungkinan-kemungkinan yang terjadi di suatu bisnis sangat diperlukan untuk zaman ini.}

\section{Perbedaan Python 2 dengan Python 3}
\usepackage{Syntax untuk mencetak teks 
Contoh nya dalam perintah print di python 2 bisa pakai tanda kurung dan bias juga tidak pakai tanda kurung. Sedangkan Python 3 harus memakai tanda kurung ini digunakan untuk mencetak 1 baris. lalu Syntax untuk meminta inputan, Hasil dari operator pembagian}

\section{Implementasi dan penggunaan python di perusahaan dunia}
\usepackage{berikut adalah 5 perusahaan dunia yang berhasil memecahkan permasalahan rumit dan membuat aplikasi top menggunakan Python.
Spotify
Penyedia layanan musik streaming Spotify memanfaatkan Python untuk analisis data dan backend. Pada backend Spotify banyak terdapat service yang berkomunikasi lewat 0MQ (ZeroMQ) yang merupakan framework dan library open source untuk networking. 0MQ dibuat menggunakan Python dan C++. Alasan service dibuat menggunakan Python dikarenakan Spotify sangat menyukai kecepatan pipeline development.
Sistem rekomendasi Spotify bergantung pada analisis data yang sangat besar, untuk menginterpretasikan analisis tersebut Spotify menggunakan Luigi, modul Python yang sinkron dengan Hadoop. Modul open source ini menangani satu library dengan library lainnya agar saling bekerjasama, dan mengkonsolidasi eror log secara cepat.
Google
Dari awal berdiri, Google sudah menggunakan Python, bahkan Python merupakan salah satu bahasa pemrograman yang penting bagi Google, itulah mengapa Google pernah merekrut kreator Python Guido Van Rossum untuk bekerja di Google.
Sebuah kutipan dari pendiri Google “Python where we can, C++ where we must,” kutipan ini artinya jika menginginkan kontrol akan memori dan latensi yang rendah maka gunakan C++, sisanya gunakan Python sebisa mungkin, meskipun ada script yang ditulis untuk Google dalam bahasa Perl atau Bash, nantinya script tersebut akan diubah lagi ke Python, alasannya adalah karena kemudahan dalam perawatan.
Saat ini Python merupakan salah satu bahasa pemrograman server-side resmi di Google, selain Pyhton Google juga menggunakan C++, Java dan Go.
Industrial Light and Magic
ILM adalah Studio spesial-efek yang didirikan oleh George Lucas pada tahun 1975, awalnya studio ini membuat berbagai efek yang dibutuhkan untuk film Star Wars saja, namun studio ini berkembang pesat dan meraih banyak penghargaan. Seiring berkembangnya teknologi komputer , ILM meyakini bahwa CGI merupakan masa depan bagi efek visual dan mulai mencari sistem yang tepat. Karena infrastruktur awal ILM dibuat menggunakan C dan C++, maka akan lebih mudah mengintegrasikan Python ketimbang Perl ataupun Tcl. Menggunakan Python, ILM dapat dengan mudah membungkus komponen software dan meningkatkan aplikasi grafis mereka. Hingga saat ini ILM tetap menggunakan Python karena selalu dapat menghadirkan solusi terbaik untuk kebutuhan mereka.
Netflix
Salah satu penggunaan utama Python di aplikasi Netflix adalah pada Central Alert Gateway. Aplikasi RESTful ini akan me-reroute alert dan mengirimkannya pada kelompok atau individu yang berhak melihatnya. Sebagai tambahan aplikasi ini akan secara otomatis reboot atau menghentikan proses yang dianggap bermasalah. Selain C.A.G, Python juga digunakan pada aplikasi untuk menelusuri riwayat dan perubahan pengaturan keamanan.


Instagram
Seperti yang kita ketahui, Instagram telah merevolusi komunikasi visual dan pemasaran digital melalui media foto. Dengan 400 juta pengguna aktif setiap harinya, tentu ini menghapus pendapat yang mengatakan bahwa aplikasi python tidak terlalu scalable. Menurut Hui Ding, engineer di Instagram, moto para pengembang aplikasi di instagram adalah “Do the simple thing first,” dan hal ini sangat bisa dilakukan menggunakan Python, bagi para pengembang aplikasi di Instagram Python sangat ramah pengguna, sederhana dan rapi. Juga karena Python sangat populer, maka tidaklah sulit menemukan pengembang baru untuk memperbesar tim.
Selain perusahaan diatas ada beberapa perusahaan pengguna Python lain yang layak untuk disebutkan yaitu, Pinterest, Disqus, Dropbox, Uber, Reddit, Quora, Facebook (Bahasa ke-3 setelah PHP (Hack) dan C++, digunakan untuk manajemen infrastruktur)
Seperti yang sudah kita ketahui, Python memang banyak digunakan oleh lingkungan akademis dan startup karena Python sederhana. Namun, seringkali solusi sederhana adalah yang terbaik. Semakin kompleks aplikasi, maka semakin terbuka kemungkinan untuk melakukan kesalahan, banyak perusahaan besar yang enggan dengan kesalahan karena akan mencederai reputasi mereka, itulah kenapa banyak perusahaan dan aplikasi besar menggunakan Python, dengan tools yang sederhana, terbukti dapat melahirkan aplikasi yang mengagumkan. Tentu saja selama programer membuat aplikasi untuk kemudahan pengguna dan membuat code sesuai dengan anjuran komunitas.

}
\end{document}
