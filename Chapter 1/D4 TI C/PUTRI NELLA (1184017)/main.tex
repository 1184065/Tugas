\documentclass{article}
\usepackage[utf8]{inputenc}

\title{Tugas Pemrograman II}
\author{Putri Nella (1184013)}
\date{19 October 2019}

\begin{document}

\maketitle
\begin{center}
    
\end{center}

\section{pengertian python}
\usepackage{Python merupakan bahasa pemrograman tingkat tinggi serta multiguna dimana perancangan berfokus pada bagaimana kode terbaca.Bahasa pemrograman Python disebut juga bahasa yang kemampuan yang dapat menggabungkan kapabilitas dengan sintaksis kode yang jelas demgam dilengkapi fungsional pustaka standar.}

{Python juga merupakan bahasa pemrograman scripting yamng memiliki tingkat tinggi,interpreted,interactive,dan berorientasi objek.Python dibentuk dengan desain yang sangat mudah dibaca dan dipahami,karena sama seperti bahasa pemrograman lainnya yaitu dengan menggunakan kata bahasa inggris.Selain itu juga lebih sedikit dalam peenggunaan rumus atau syntac}
\section{Sejarah Python}
\usepackage{Sejarah penemuan python berawal dari orang keturunan belanda yaitu Guido van Rossum.Pembuatan bahasa pemrograman python ini berlangsung di kota amsterdam Belanda pada tahun 1995.Pyhthon dikembangkan ladi agar lebih kompatibel oleh Goido Van Rossum.Kemudian pada awal tahun 2000,terdapat pembaharuan vwrsi Python hingga mencapai versi 3 sampai saat ini.Nama Python sendiri diambil dari sebuah acara televisi yang lumayan terkenal yang bernama Mothy Python Flying Circus yang merupakan acara sirkus favorit dari Goido Van Rossum.}
\section{Perbedaan Python 2 dan 3}
\usepackage{Ada dua python yang popular saat ini,yaitu python versi 2 dan python versi 3,berikut adalah perbedaan python versi 2 dan 3.}
\subsection{Phython versi 2}
\usepackage{dipublikasikan pada akhir tahun 2000, Python 2 dinilai akan lebih transparan dan inklusif untuk pengembangan software dibandingkan versi sebelumnya.Ini didukung dengan adanya PEP – Python Enhancement Proposal,yang merupakan spesifikasi teknis menjadi tuntunan informasi untuk penggunanya dan menggambarkan fitur baru di Python itu sendiri.
Python 2 juga dilengkapi dengan berbagai fitur programatikal seperti cycle-detecting garbage collector untuk mengotomasi manajemen memori, peningkatan dukungan untuk Unicode, list comprehension untuk membuat sebuah list berdasarkan list yang sudah ada. Unifikasi pada tipe data Python dan class ke satu hirarki terjadi pada rilis Python 2.2}
\subsection{Python versi 3}
\usepackage{Python 3 ini diharapkan sebagai Python masa depan dan merupakan versi yang tulisan yang akan dibuat ini  masih aktif dikembangkan. Python 3 sendiri merupakan versi dengan banyak perubahan/modifikasi yang telah dirilis di akhir tahun 2008.Python 3 sendiri fokus untuk melakukan perapian pada codebase dan menghapuskan duplikasi (redundancy). Perubahan terbesar pada Python 3 termasuk memasukkan statemen print ke dalam built-in function.}
\section{Implementasi dan Penggunaan Python di Perusahaan Dunia}
\usepackage{Membicarakan bahasa pemrograman yang populer, tampaknya kita tak bisa lepas dari bahasa pemrograman Python.Bahasa pemrograman python pun kini telah dikenal oleh developer Indonesia. Bahasa Pemrograman Python, kini bahkan menempati posisi keempat bahasa pemrograman populer versi Tiobe Index.}
\subsection{Implementasi}
\usepackage{kali saya akan mengambil contoh beberapa platform yang menggunakan python dalam alisis data yaitu instagram dan pinterest sebagai berikut }
\begin{enumerate}
    \item Instagram, yang pertama adalah instagram,siapa yang tidak mengenal instagram menggunakan Django, framework python sebagai mesin pengolah sisi server dari aplikasinya.
    \item Pinterest merupakan platform lain untuk mengekspresikan ide kreatif kita.kita dapat menemukan banyak informasi dalam bentuk gambar, poster, info-grafis, yang merupakan isi utama dari Pinterest. Untuk mengelola konten dari kategori ini, Pinterest mengandalkan bahasa pemrograman Python.
\end{enumerate}
\section{Instalasi}
\subsection {Instalasi Anaconda}
\begin{itemize}
    \item langkah awal adalah mengunjungi website (http://www.anaconda.com/distribution/#downloadsection),download python sesuai dengan sistem operasi yang dipakai.
    \item buka file installer,klik "next"
    \item klik "i agree"
    \item klik "next"
    \item tentukan lokasi folder anaconda yang akan di install lalu klik "next"
    \item klik centang pada "ADD anaconda to my PATH enviroment variable" lalu klik "next".
    \item Tunggu hingga proses selesai lalu klik "next".
    \item klik "next"
    \item klik "finish"
\end{itemize}
\subsection {Intalasi Python}
\begin{itemize}
    \item Langkah awal kunjungi dulu website python,download sesuai pilihan user, rekomendasi versi 3.
    \item Pilih install now, tapi jangan lupa tambahan centang untuk add path.
    \item Tunggu sebentar.
    \item Akhirnya selesai.
\end{itemize}
\subsection {Instalasi Pip}
\begin{itemize}
    \item Ketikan Pip di CMD.
\end{itemize}
\subsection{Setting Environment}
\begin{itemize}
    \item Buka CMD,ketik "python" apabila dikenali maka tidak perlu setting environment variable
\end{itemize}
\subsection{ mencoba entrepreter/cli di cmd windows}
\begin{itemize}
     \item Pertama, kita coba dulu membuka Python Shell. Silahkan masuk ke manu kemudian cari Python Shell.
    \item dari CMD, ketik perintah python untuk masuk ke Python Shell dari CMD,
    \item Ketikkan perintah di IDLE.
\end{itemize}
\subsection{Print Hello World}
\begin{itemize}
	\item Buka spyder
	\item tuliskan syntax print("hello world")
	\item klik run script atau tekan F5
	\item tuliskan nama file lalu klik save.
\end{itemize}
\subsection{Variable exploler}
\paragraph{}
Variabel exploler digunakan sebagai pengecekan semua variabel yang punya value.
\subsection{ Cara menjalankan Script otomatis login aplikasi akademik dengan library selenium dan inputan user}
\begin{itemize}
	\item \textit{from selenium import webdriver}, untuk import webdriver
	\item \textit{driver = webdriver.Firefox()},  untuk memilih webdriver firefox
	\item \textit{driver.maximize\_window()},  untuk maximize window browser
	\item \textit{driver.get("http://siap.poltekpos.ac.id/siap/besan.depan.php")}, untuk redirect ke website tujuan
	\item \textit{driver.find\_element\_by\_name('user\_name').send\_keys('1184063')}, untuk mengisi form username, driver akan mencari element dengan nama user\_name
	\item \textit{driver.find\_element\_by\_name('user\_pass').send\_keys('pwnyadisiniya')}, untuk mengisi form password,lalu driver akan mencari element dengan nama pass\_name
	\item \textit{driver.find\_element\_by\_name('login').click()}, untuk klik tombol login, dengan fungsi click()
	
\section{Indentasi}
\subsection{Penjelasan Identasi }
\usepackage{Indentasi merupakan suatu cara perapihan sintaks atau sebagi aturan dalam Bahasa pemrogaman yang akan ditulis. Indentasi digunakan untuk acuan scope pemrograman dan compiler seperti Bahasa pemrograman python. Indentasi ditandai atau berkaitan dengan kurung kurawal ‘’ untuk memulai atau mengakhiri suatu scope permasalahan. Indentasi sering menjadi suatu kebiasaan atau khas dari seorang programmer. Biasanya indentasi dipakai untuk sekedar memudahkan pembacaan kode program, namun dalam Python,Fungsi indentasi sebagai penanda blok kode program.}

\subsection{ jenis jenis error identasi yang didapat}
\usepackage{script pada spyder tidak sesuai posisi, contohnya penulisan script terlalu menjorok atau penulisan yg kurang menjorok ke dalam }

\subsection{Cara membaca eror}
\usepackage{ untuk mengetahui jenis errornya dapat diketahui melalui IndentitionError: expected an idented block. untuk menghindari error ini bisa menggunakan fungsi if memerlukan identasi untuk membedakannya.}
\end{itemize}
\section{Cara Menangani Eror}
\usepackage{IndentationError: unindent does not match any outer indentation level}

\begin{itemize}
    \item Cara mengatasi eror tersebut, solusinya adalah kita harus mengecek penggunaan tab / spasi dengan konsisten. tapi menurut aturan PEP-8 menyarankan kita menggunakan 4 spasi untuk satu level indentation.
\end{itemize}

\end{document}
