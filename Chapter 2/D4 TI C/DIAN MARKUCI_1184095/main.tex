\documentclass{article}
\usepackage[utf8]{inputenc}

\title{Tugas Pemrograman Chapter 2}
\author{Dian Markuci(1184095) }
\date{25 October 2019}

\begin{document}

\maketitle

\section{TEORI}
\subsection{Variable}
\usepackage{Variabel merupakan suatu tempat untuk menampung value dimemori, ibarat sebuah ruangan, variabel dibagi menjadi dua berdasar ruang lingkup yaitu variable global dan lokal, untuk menentukan variabel global atau lokal, tergantung dari tempat dideklarasikannya variabel pada program yang sedang dibangun. Variabel global adalah variabel yang bisa diakses di semua lingkup  dalam program yang sedang dibuat, atau disebut variabel global ini bisa dikenali oleh semua fungsi dan prosedur, sementara variabel lokal adalah variabel yang dapat diakses hanya di lingkup khusus,atau variabel lokal ini hanya bisa diakses pada fungsi/prosedur dimana variabel itu dideklarasikan.}

\subsection{Input dan Output User}
\begin{itemize}
    \item Input menggunakan fungsi input ()
    \item Python memiliki fungsi input yang memungkinkan Anda meminta input teks kepada pengguna. Anda memanggil fungsi ini untuk memberi tahu program untuk berhenti dan menunggu pengguna memasukkan data
    \item Pada Python 2, Anda memiliki fungsi built\textunderscore in raw\textunderscore input ()
    \item Pada Python 3, Anda memiliki input ().\\
    
\end{itemize}

\subsection{Operator 2}
\subsubsection{Operasi Aritmatika}

\begin{tabular}{|c|c|}
\hline
Operator & Simbol\\
\hline
Pembagian & /\\
\hline
Perkalian & *\\
\hline
Penjumlahan & +\\
\hline
Pengurangan & -\\
\hline
Modulus & \% \\
\hline
Pangkat & **\\
\hline
\end{tabular}

\paragraph{}
Operasi matematika sebagai bahasa pemrograman, Python memiliki operasi aritmatika seperti tambah, kurang, kali, bagi. Berikut adalah contoh penggunaan operasi aritmatika pada python
Contoh misalnya kita mempunyai variable : a=6 dan b=4\\

\subsubsection{Casting}
\paragraph{}
Lalu apa yang akan terjadi bila ternyata variabel a adalah string dan variabel b adalah integer, contoh a=”6” dan b=4, tentunya program akan error bukan? Disinilah casting digunakan. Casting adalah cara untuk mengubah tipe data dari suatu data primitive, Jadi misal kita akan menjumlahkan variabel a dan b seperti contoh diatas tetapi logikanya sebuah kata (string) tidak akan bisa dijumlahkan dengan angka (“6” + 4) karena variabel a diapit oleh tanda kutip, ini berarti variabel a bertipe data string untuk itu kita perlu merubah dulu variabel a yang tadinya string menjadi integer. Berikut adalah sintax untuk melakukan casting :
\begin{itemize}
	\item int(var/value) : mengubah tipe data ke integer, contoh int(angka)
	\item float(var/value) : mengubah tipe data ke float, contoh float(hasil)
	\item string(var/value) : mengubah tipe data ke str, contoh string(12)
\end{itemize}
Mengubah string ke integer : type data string harus dilakukan casting dengan "int(variable)".
Mengubah integer ke string : type data integer harus dilakukan casting dengan "str(variable)".\\

\subsection{Syntax Perulangan}

\item Di dalam bahasa pemrograman Python pengulangan dibagi menjadi 3 bagian

\begin{itemize}
    \item While Loop
    \item For Loop
    \item Nested Loop
\end{itemize}

\item \textbf{While Loop} \\ 
While : untuk melakukan looping yang tidak pasti\\
Contoh :\\
i = 0\\
while True :\\
    if i \textless 100:\\
        print ("i bernilai : "), i\\
        i = i + 1\\
    elif i \textgreater = 100:\\
        break\\
        
\item \textbf{For Loop}\\
For : Melakukan looping yang sudah pasti jumlahnya\\
Contoh :\\
for i in range(0, 100):\\
    print (i)\\
    
\item \textbf{Nested Loop}

\item Bahasa pemrograman Python memungkinkan penggunaan satu lingkaran di dalam loop lain. Dibawah ini menunjukkan beberapa contoh untuk menggambarkan konsep tersebut

#Contoh penggunaan Nested Loop

i = 2\\
while(i \textless 100):\\
    j = 2\\
    while(j \textless= (i/j)):\\
        if not(i%j): break\\
        j = j + 1\\
    if (j \textgreater i/j) : print(i, " is prime")\\
    i = i + 1\\

print "Good bye!"

    \subsection {Syntax Kondisi}\\
    \textbf{Struktur if}
    \item Sederhananya struktuf if dalam Python dijalankan untuk mengecek kondisi ini bernilai benar atau salah. Apabila kondisi  bernilai benar, maka python akan menjalankan statement dalam blok kondisi tersebut dan sebaliknya jika kondisi bernilai salah maka statement dalam blok tersebut tidak akan dijalankan.\\
    
    \textbf{Struktur if – else}
    \item Struktur if sebelumnya hanya menjalankan statement dalam blok kondisi jika bernilai benar, maka struktur if-else adalah membuat statement kondisi yang bernilai benar dan salah.


\item If : digunakan untuk percabangan\\
Contoh :\\
umur = 20\\
if umur \textgreater 17:\\
    print("Remaja")\\
\item Else\\
Else : jika kondisi if tidak terpenuhi maka dijalankan kondisi else\\
Contoh :\\
umur = 6\\
if umur \textgreater 17:\\
    print ("beranjak dewasa")\\
else:\\
    print ("anak-anak")\\

\subsection{Try & Except}\\
\textbf{Menangani Eksepsi Menggunakan Try, Except, dan Finally}\\

Terjadinya eksepsi pada suatu program bisa membuat program berhenti. Untuk mencegahnya, Maka kita harus mengantisipasi hal tersebut.

Python menyediakan metode penanganan eksepsi dengan menggunakan pernyataan \textit{try dan except}.

Dalam blok try kita akan meletakkan baris program yang mungkin akan terjadi error. Apabila terjadi error, maka cara penanganannya diserahkan kepada blok except. \\

\textbf{contoh try…finally untuk mengoperasikan file.} 
try:\\
    f = open("C:test.txt")\\
    # melakukan operasi terhadap file\\
finally:\\
    f.close()\\






\end{document}
