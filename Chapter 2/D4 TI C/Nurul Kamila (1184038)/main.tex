\documentclass{article}
\usepackage[utf8]{inputenc}

\title{Chapter2}
\author{nurulkamila1899 }
\date{October 2019}

\begin{document}

\maketitle

\section{Jenis-Jenis Variabel dan Cara Pemakaian Variabel}
Variabel merupakan suatu wadah untuk menyimpan data, sedangkan data yang disimpan di dalam variabel disebut tipe data. Tipe data terbagi menjadi beberapa macam, yaitu:
\subsection{Tipe Data Angka}
Tipe ini terbagi menjadi beberapa jenis lagi, yaitu:
\subsubsection{Integer (bilangan bulat)}
Contoh : 1,2,3,dst.
\subsubsection{Float (bilangan pecahan)}
contoh : 1.5, 2.1, dst.
\subsection{Tipe Data Teks}
Tipe ini terbagi menjadi dua jenis, yaitu:
\subsubsection{String (kumpulan karakter)}
Contoh : ”Saya sedang makan”
\subsubsection{Varchar (karakter)}
Contoh : a,b,c,dst.\\
Penulisan tipe ini harus diapit oleh tanda petik.
\subsection{Tipe Data Boolean}
Tipe ini adalah tipe yang hanya memiliki dua nilai yaitu True dan False atau 0 dan 1.
\section{Kode Untuk Meminta Input Dari User dan Melakukan Output ke Layar Input }
Kode Untuk Meminta Input Dari User dan Melakukan Output ke Layar
Input merupakan suatu masukan yang akan kita berikan ke program. Sedangkan hasil yang ditampilkan disebut output.
\subsection{Cara mengambil input}
Python menyediakan fungsi input, menggunakan kode input()
\subsection{Cara menampilkan output}
Untuk menampilkan sebuah ouput teks kita menggunakan kode print()
\section{Operator Dasar Aritmatika, Tambah, Kali, Kurang, Bagi, dan Bagaimana Mengubah String ke Integer dan Integer Ke String}
Operator merupakan suatu simbol-simbol yang digunakan untuk operasi tertentu. Dan operator aritmatika termasuk pada operator yang paling sering digunakan.
\section{Penjelasan Syntax Untuk Perulangan dan Jenis-Jenisnya serta Contoh Kode dan Cara Menggunakannya di Python}
Perulangan berfungsi untuk melakukan sesuatu secara berulang atau terus menerus. Pada bahasa pemrograman terdapat dua jenis perulangan, yaitu For dan While. Perulangan For disebut juga perulangan yang terhitung (counted loop) sedangkan perulangan while disebut perulangan yang tak terhitung (uncounted loop). Secara umum python mengeksekusi program secara berbaris, tetapi untuk perulangan satu baris dieksekusi beberapa kali. Perulangan memerlukan tes kondisi, jika hasil tes true maka blok tersebut akan terus dieksekusi sedangkan jika false maka akan keluar dari blok perulangan dan mengeksekusi blok selanjutnya.
\subsection{Perulangan For}
Perulangan ini biasanya digunakan untuk mengetahui kode yang sudah banyak perulangannya. Adapun contoh kodenya adalah:\\
\\
ulang = 5
 for i in ( ulang ) : 
 print (”Perulangan ke−”+str ( i ) )
\subsection{Perulangan While}
Bila kondisi yang diuji salah, maka loop tidak akan pernah dieksekusi\\
\\
 umur = 19 
 i f umur > 17: 
 print (”Anda sudah cukup umur untuk membuat KTP”)
\section{Cara Menggunakan Syntax Untuk Memilih Kondisi dan Contoh Syntax Kondisi Di Dalam Kondisi}
Python memiliki tiga jenis kondisional yang dapat digunakan untuk membangun suatu alur logika. Yaitu if, ifelse, dan ifelifelse.
\subsection{Kondisi If}
Jika kondisi utama true, maka perintah akan dijalankan
\subsection{Kondisi If Else}
Untuk memeriksa kondisi utama, else digunakan untuk menangani kondisi selain kondisi yang telah ditentukan.
\subsection{Kondisi if elif else}
Bila anda akan mendefinisikan cukup banyak kondisi, maka gunakan elif di bawah statement if dan diatas statement else.
\subsection{If di dalam if (If bersarang)}
Suatu kondisional dapat disimpan di dalam if lain
\section{Jenis error yang sering ditemui di Python dalam mengerjakan Syntax }
Biasanya error terjadi dikarenakan ada kesalahan dalam pengetikan syntax, jika nilai bertipe data integer maka harus menggunakan range setelah in.
\section{Cara Memakai Try Except}
Try except biasa digunakan untuk menangani error saat penggunaan IO, database, atau pengaksesan indeks suatu list atau dictionary, dll.

\end{document}
