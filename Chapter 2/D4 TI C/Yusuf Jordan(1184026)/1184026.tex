\documentclass[12pt]{ociamthesis}  % default square logo 
%\documentclass[12pt,beltcrest]{ociamthesis} % use old belt crest logo
%\documentclass[12pt,shieldcrest]{ociamthesis} % use older shield crest logo

%load any additional packages
\usepackage{amssymb}
\usepackage{listings}
\usepackage{float}
\usepackage{graphics}
\usepackage{color}
 
\definecolor{codegreen}{rgb}{0,0.6,0}
\definecolor{codegray}{rgb}{0.5,0.5,0.5}
\definecolor{codepurple}{rgb}{0.58,0,0.82}
\definecolor{backcolour}{rgb}{0.95,0.95,0.92}
 
\lstdefinestyle{mystyle}{
    backgroundcolor=\color{backcolour},   
    commentstyle=\color{codegreen},
    keywordstyle=\color{magenta},
    numberstyle=\tiny\color{codegray},
    stringstyle=\color{codepurple},
    basicstyle=\footnotesize,
    breakatwhitespace=false,         
    breaklines=true,                 
    captionpos=b,                    
    keepspaces=true,                 
    numbers=left,                    
    numbersep=5pt,                  
    showspaces=false,                
    showstringspaces=false,
    showtabs=false,                  
    tabsize=2,
    language=python
}
 
\lstset{style=mystyle}

%input macros (i.e. write your own macros file called mymacros.tex 
%and uncomment the next line)
%\include{mymacros}

\title{Tugas Chapter 2\\[1ex]     %your thesis title,
        Pemrograman II}   %note \\[1ex] is a line break in the title

\author{Yusuf Jordan El Anwar}             %your name
\college{1184026\\[5ex]
Applied Bachelor of Informatics Engineering}  %your college

%\renewcommand{\submittedtext}{change the default text here if needed}
\degree{Politeknik Pos Indonesia}     %the degree
\degreedate{Bandung 2019}         %the degree date

%end the preamble and start the document
\begin{document}

%this baselineskip gives sufficient line spacing for an examiner to easily
%markup the thesis with comments
\baselineskip=18pt plus1pt

%set the number of sectioning levels that get number and appear in the contents
\setcounter{secnumdepth}{3}
\setcounter{tocdepth}{3}


\maketitle                  % create a title page from the preamble info
\begin{dedication}
`InsyaAllah Menang, Kelompok terdebes? Jangan Ditanya, Pasti Kelompok 2 '\\ 
\end{dedication}        % include a dedication.tex file

\begin{romanpages}          % start roman page numbering
				
\end{romanpages}            % end roman page numbering



%now include the files of latex for each of the chapters etc
\chapter{Mengenal Python dan Anaconda}
Tujuan pembelajaran pada pertemuan pertama antara lain:
\begin{enumerate}
\item
Mengerti sejarah python, perkembangan dan penggunaan python di perusahaan
\item
Memahami tahapan instalasi python dan anaconda
\item
Memahami cara penggunaan spyder
\end{enumerate}
Tugas dengan cara dikumpulkan dengan pull request ke github dengan menggunakan format latex pada repo yang dibuat oleh asisten IRC.

\section{Teori}
Praktek teori penunjang yang dikerjakan :
\begin{enumerate}
\item
Buat Resume Sejarah Python, perbedaan python 2 dan 3, dengan bahasa yang mudah dipahami dan dimengerti. Buatan sendiri bebas plagiat(10)
\item
Buat Resume Implementasi dan penggunaan Python di perusahaan dunia, bahasa yang mudah dipahami(10)
\end{enumerate}

\section{Instalasi}
Melakukan instalasi python dan anaconda versi 3 serta uji coba spyder. Dengan menggunakan bahasa yang mudah dimengerti dan bebas plagiat. 
Dan wajib skrinsut dari komputer sendiri.
\begin{enumerate}
\item
Instalasi python 3 (5)
\item
instalasi pip(5)
\item
cara setting environment (5)
\item
mencoba entrepreter/cli melakui terminal atau cmd windows(5)
\item 
Menjalankan dan mengupdate anaconda dan spyder(5)
\item
Cara menjalankan Script hello word di spyder(5)
\item
Cara menjalankan Script otomatis login aplikasi akademik dengan library selenium dan inputan user(5)
\item
Cara pemakaian variable explorer di spyder(5)
\end{enumerate}


\section{Identasi}
Membuat file main.py dan mengisinya dengan script contoh python penggunaan selenium(minimal 20 baris) yang melibatkan inputan user, kemudian mencoba untuk mengatasi error identasi.
\begin{enumerate}
	\item
Penjelasan Identasi (10)
	\item
jenis jenis error identasi yang didapat(10)
\item
cara membaca error(10)
\item 
cara menangani errornya(10)
\end{enumerate}

\section{Presentasi Tugas}
Pada pertemuan ini, diadakan tiga penilaiain yaitu penilaian untuk tugas mingguan dengan nilai maksimal 100. Kemudian dalam satu minggu kedepan maksimal sebelum waktu mata kuliah. Ada presentasi kematerian dengan nilai presentasi yang terpisah masing-masing 100. Dan nilai terpisah untuk tutorial dari jawaban tugas di YouTube.Jadi ada tiga komponen penilaiain pada pertemuan ini yaitu :
\begin{enumerate}
	\item tugas minggu hari ini dan besok (maks 100). pada chapter ini
	\item presentasi csv (maks 100). Mempraktekkan kode python dan menjelaskan cara kerjanya.
	\item pembuatan video tutorial youtube tentang tutorial dari jawaban tugas.(nilai maks 100)
\end{enumerate}
Waktu presentasi pada jam kerja di IRC. Kriteria penilaian presentasi sangat sederhana, presenter akan ditanyai 20(10 pertanyaan program, 10 pertanyaan teori) pertanyaan tentang pemahamannya menggunakan python dan program agan dibuat error hingga presenter bisa menyelesaikan errornya. jika presenter tidak bisa menjawab satu pertanyaan asisten maka nilai nol. Jika semua pertanyaan bisa dijawab maka nilai 100. Presentasi bisa diulang apabila gagal, sampai bisa mendapatkan nilai 100 dalam waktu satu minggu kedepan.

\chapter*{Perusahaan Dunia yang menggunakan bahasa pemrograman Python}

\section*{\textit{Spotify}}
\par
\textit{Spotify} merupakan layanan musik streaming yang sudah banyak digunakan di seluruh dunia. Dalam bidang menganalisis data spotify menggunakan Bahasa Pemrograman Python, dalam pengimplementasiannya Tim Spotify menggunakan Luigi, modul yang ada di Python yang disingkronisasikan pada sebuah software yang memudahkan programmer membuat aplikas web atau disebut framework yang berbasis Java yang memungkinkan pemrosesan data dalam waktu cepat.Penerapan bahasa Python juga digunakan dalam penerapan fitur Radio dan Discover serta fitur merekomendasikan orang yang mungkin akan diikuti.

\section*{\textit{Netflix}}
\par
\textit{Netflix} merupakan layanan pemutaran film atau tayangan yang memungkinkan para penggunaknya menggunakan di manapun dan kapanpun. Netflix menggunakan bahasa pemrograman Python. Penggunaan Python di Netflix terdapat pada Central Alert Gateaway (C.A.G) ini akan me-reroute alert dan mengirimkannya pada kelompok atau individu yang dapat melihatnya dan secara otomatis reboot atau menghentikan proses yang dianggap bermasalah dan digunakan untuk menulusuri riwayat dan perubahan pengaturan keamanan. Tetapi sama halnya seperti spotify, Netflix menggunakan python untuk menganilisis data dan lebih utama terlihat pada bagian bagaimana netflix merekomendasikan film kepada pelangganya.

\section*{\textit{Pinterest}}
\par
\textit{Pinterest} adalah aplikasi web yang digunakan untuk mengumpulkan hal-hal yang menarik berdasarkan kriteria tertentu yang sering dikunjungi di jaman sekarang. Pinterest menggunakan Bahasa Pemrograman Python dari awal mereka membangunnya itulah sebabnya bookmarking (Sebuah metode bagi pengguna internet untuk mengorganisasi, menyimpan, mengelola, dan mecari penanda sumber daya yang tersedia secara online) yang ada di pinterest begitu terstruktur dan mudah untuk diatur.

\section*{\textit{Instagram}}
\par
\textit{Instagram} merupakan sebuah aplikasi berbagi foto dan video secara digital yang digunakan oleh lebih dari 400 juta user yang aktif setiap harinya. Instagram menggunakan bahasa pemrograman python dalam task queuenya atau fitur dimana pada saat yang bersamaan instagram dapat melakukan posting ke beberapa social network lainnya seperti Facebook, Twitter, dll. 

\section*{\textit{Industrial Light and Magic}}
\textit{Indusrial Light and Magic} adalah Studio spesial-efek yang digunakan pada pemutaran efek di film Star Wars. Dalam pembuatan efek ledakan ILM menggunakan Python dikarenakan dapat menghemat waktu dalam pembuatan efek tersebut.   
\par


\documentclass[a4paper,12pt]{report}
\usepackage{listings}
\title{Tugas Chapter 3}
\author{Rayhan Yuda Lesmana}
\date{27 Oktober 2019}

\begin{document}
\maketitle
\chapter*{Teori}
\section*{Fungsi}
\paragraph{}
Fungsi dalam python merupakan satu blok program yang terdiri dari nama fungsi, input variabel, dan variabel kembalian. Nama fungsi di python diawali dengan def dan setelahnya tanda titik dua. Fungsi input() dan raw\_input digunakan untuk mengambil data angka dan teks.Cara mengembalikan nilai dari sebuah fungsi dengan  menggunkan kata kunci return lalu diikuti dengan nilai atau variabel yang akan dikembalikan.\\
\paragraph{}
Inputan fungsi(parameter) adalah inputan sebuah fungsi bertujuan menyimpan sebuah nilai.\\
\paragraph{}
kembalian fungsi(return) berfungsi untuk mengembalikan sebuah nilai, dan bisa juga mengakhiri sebuah eksekusi fungsi.\\
Contoh syntax:\\
def function(a,b):\\
	c=a*b\\
	return c\\
\section*{Paket dan Library}
\paragraph{}
Package adalah folder yang menyimpan syntax code. Semisal kita membuat folder yang di dalam nya terdapat syntax program kita dengan nama remote.py.\\
Cara memanggil package:\\
from remote import batre
 
\section*{Kelas, Objek, Atribut dan methode}
\paragraph{Class}
Class adalah cetak biru (blueprint) dari sebuah onject.\\
Contoh syntax:\\
class Name:
	def \_ \_init\_ \_(self,name):
		self.name = name
	def hayname(self):
		print("Haii", name)	
\paragraph{Obejct}
Obejct adalah hasil cetak dari sebuah class.\\
Contoh syntax:\\
import kelas3lib\\
\\
cobakelas=kelas3lib.aku(npm)\\
hasil=cobakelas.NPM2()\\
\paragraph{Atribut}
Atribut adalah variabel yang dimiliki olleh sebuah class.\\
Contoh syntax:\\
class Name:\\
def \_ \_init\_ \_(self,nama):\\
		self.nama = nama\\
\paragraph{Method}
Method fungsi dari sebuah class.\\
Contoh syntax:\\
class Name:\\
	def \_ \_init\_ \_(seld,name):\\
		self.name = name\\
	def name(self):\\
		print("hayy",name)\\
\section*{Library}
\paragraph{}
Membuat folder library dengan nama try:\\
def Name():\\
	print("Rayhan yuda")\\
Contoh memanggil fungsi dari library:\\
import try\\
try.Name()\\
\paragraph{Penggunaan package}
from kalkulator import perkalian\\
Contoh lainnya:\\
from motor import bensin\\
\paragraph{Pemanggilan library dalam folder}
Contoh syntax:\\
from motor import bensin\\
Memakai library bensin.\\
\paragraph{Pemanggilan class dalam folder}
Contoh syntax:\\
from motor import bensin\\
Memakai class bensin.\\
\chapter*{Keterampilan pemograman}
\begin{enumerate}
\item Question1
\lstinputlisting[language=Python]{src/NPM1.py}
\item Question2
\lstinputlisting[language=Python]{src/NPM2.py}
\item Question3
\lstinputlisting[language=Python]{src/NPM3.py}
\item Question4
\lstinputlisting[language=Python]{src/NPM4.py}
\item Question5
\lstinputlisting[language=Python]{src/NPM5.py}
\item Question6
\lstinputlisting[language=Python]{src/NPM6.py}
\item Question7
\lstinputlisting[language=Python]{src/NPM7.py}
\item Question8
\lstinputlisting[language=Python]{src/NPM8.py}
\item Question9
\lstinputlisting[language=Python]{src/NPM9.py}
\item Question10
\lstinputlisting[language=Python]{src/NPM10.py}
\item Question11
\lstinputlisting[language=Python]{src/lib3.py}
\item Question12
\lstinputlisting[language=Python]{src/kelas3lib.py}
\end{enumerate}
\chapter*{Keterampilan penanganan error}
\paragraph{Penanganan error}
error:\\
Tipe error: \_ \_ init\_ \_ missing 1 required positional argument: "npm"\\
Penyelesaian:\\
Menambahkan para meter.\\

Syntax:\\
def perkalian(a,b):
	c=a*b
	return c
	
d=int(input("Angka"))
e=int(input("Angka"))
try:
	print(perkalian(d,e))
except:
	print("Tidak boleh 0")		
\end{document}

%now enable appendix numbering format and include any appendices
\appendix


%next line adds the Bibliography to the contents page

\end{document}

