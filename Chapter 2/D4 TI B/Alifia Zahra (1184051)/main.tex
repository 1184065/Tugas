\documentclass[12pt]{ociamthesis}  % default square logo 
%\documentclass[12pt,beltcrest]{ociamthesis} % use old belt crest logo
%\documentclass[12pt,shieldcrest]{ociamthesis} % use older shield crest logo

%load any additional packages
\usepackage{amssymb}
\usepackage{listings}

\usepackage{color}
 
\definecolor{codegreen}{rgb}{0,0.6,0}
\definecolor{codegray}{rgb}{0.5,0.5,0.5}
\definecolor{codepurple}{rgb}{0.58,0,0.82}
\definecolor{backcolour}{rgb}{0.95,0.95,0.92}
 
\lstdefinestyle{mystyle}{
    backgroundcolor=\color{backcolour},   
    commentstyle=\color{codegreen},
    keywordstyle=\color{magenta},
    numberstyle=\tiny\color{codegray},
    stringstyle=\color{codepurple},
    basicstyle=\footnotesize,
    breakatwhitespace=false,         
    breaklines=true,                 
    captionpos=b,                    
    keepspaces=true,                 
    numbers=left,                    
    numbersep=5pt,                  
    showspaces=false,                
    showstringspaces=false,
    showtabs=false,                  
    tabsize=2,
    language=python
}
 
\lstset{style=mystyle}

%input macros (i.e. write your own macros file called mymacros.tex 
%and uncomment the next line)
%\include{mymacros}

\title{Tugas Chapter 2 \\[1ex]     %your thesis title,
        Pemrograman II}   %note \\[1ex] is a line break in the title

\author{Alifia Zahra}             %your name
\college{1184051\\[5ex]
D4 Teknik Informatika 2B}  %your college

%\renewcommand{\submittedtext}{change the default text here if needed}
\degree{Politeknik Pos Indonesia}     %the degree
\degreedate{Bandung 2019}         %the degree date

%end the preamble and start the document
\begin{document}

%this baselineskip gives sufficient line spacing for an examiner to easily
%markup the thesis with comments
\baselineskip=18pt plus1pt

%set the number of sectioning levels that get number and appear in the contents
\setcounter{secnumdepth}{3}
\setcounter{tocdepth}{3}


\maketitle                  % create a title page from the preamble info
\begin{dedication}
`InsyaAllah Menang, Kelompok terdebes? Jangan Ditanya, Pasti Kelompok 2 '\\ 
\end{dedication}        % include a dedication.tex file
\begin{acknowledgements}
Assalamualaikum warahmatullahi wabarakatuh. Segala puji bagi Allah SWT yang telah memberikan kemudahan sehingga dapat menyelesaikan laporan Internship II ini, tanpa pertolongan-Nya mungkin penulis tidak akan sanggup menyelesaikannya dengan baik. Shalawat dan salam semoga terlimpah curahkan kepada Nabi Muhammad SAW beserta sahabat dan keluarga Beliau.

Laporan ini disusun untuk memenuhi kelulusan matakuliah Internship II pada Program Studi DIV Teknik Informatika. Proses Internship II ini juga tidak terlepas dari bantuan berbagai pihak. Oleh karena itu, pada kata pengantar ini penulis menyampaikan teriamakasih kepada :
\begin{enumerate}

\item Syafrial Fachri Pane, S.T., M.T.I. selaku Pembimbing Internal dan Penguji Utama dalam penyusunan laporan Internship II ini;
\item Krisna Sukrisna Yanuar selaku Pembimbing Eksternal dalam penyusunan laporan Internship II ini;
\item	M. Harry K Saputra, S.T., M.T.I. selaku Koordinator Internship II Tahun Akademik 2018/2019;
\item	M. Yusril Helmi Setyawan, S.Kom., M.Kom. selaku Ketua Program Studi DIV Teknik Informatika Tahun Akademik 2018/2019;
\item	Dr. Ir. Agus Purnomo, M.T. selaku Direktur Politeknik Pos Indonesia Tahun Akademik 2018/2019.

\end{enumerate}

Penulis telah membuat laporan ini dengan sebaik-baiknya, diharapkan memberikan kritik dan saran dari semua pihak yang bersifat membangun, terimakasih.

\begin{raggedleft}

Bandung, 22 Maret 2019

Penulis

\end{raggedleft}

\end{acknowledgements}   % include an acknowledgements.tex file
\begin{abstract}
	Modul Praktikum ini dibuat dengan tujuan memberikan acuan, bagi mahasiswa dan dosen
	Pengajar Mata Kuliah. Pada intinya buku ini menjelaskan secara lengkap tentang Standar penilian mata kuliah pemrograman II
	di Program Studi D4 Teknik Informatika, dan juga mengatur mekanisme, teknik penulisan, serta
	penilaiannya.Dengan demikian diharapkan semua pihak yang terlibat dalam aktivitas belajar dan mengajar
	berjalan lancar dan sesuai dengan standar.
\end{abstract}          % include the abstract

\begin{romanpages}          % start roman page numbering
\tableofcontents            % generate and include a table of contents
\listoffigures              % generate and include a list of figures
\end{romanpages}            % end roman page numbering

%now include the files of latex for each of the chapters etc
%include{section/chapter1}
\chapter*{Perusahaan Dunia yang menggunakan bahasa pemrograman Python}

\section*{\textit{Spotify}}
\par
\textit{Spotify} merupakan layanan musik streaming yang sudah banyak digunakan di seluruh dunia. Dalam bidang menganalisis data spotify menggunakan Bahasa Pemrograman Python, dalam pengimplementasiannya Tim Spotify menggunakan Luigi, modul yang ada di Python yang disingkronisasikan pada sebuah software yang memudahkan programmer membuat aplikas web atau disebut framework yang berbasis Java yang memungkinkan pemrosesan data dalam waktu cepat.Penerapan bahasa Python juga digunakan dalam penerapan fitur Radio dan Discover serta fitur merekomendasikan orang yang mungkin akan diikuti.

\section*{\textit{Netflix}}
\par
\textit{Netflix} merupakan layanan pemutaran film atau tayangan yang memungkinkan para penggunaknya menggunakan di manapun dan kapanpun. Netflix menggunakan bahasa pemrograman Python. Penggunaan Python di Netflix terdapat pada Central Alert Gateaway (C.A.G) ini akan me-reroute alert dan mengirimkannya pada kelompok atau individu yang dapat melihatnya dan secara otomatis reboot atau menghentikan proses yang dianggap bermasalah dan digunakan untuk menulusuri riwayat dan perubahan pengaturan keamanan. Tetapi sama halnya seperti spotify, Netflix menggunakan python untuk menganilisis data dan lebih utama terlihat pada bagian bagaimana netflix merekomendasikan film kepada pelangganya.

\section*{\textit{Pinterest}}
\par
\textit{Pinterest} adalah aplikasi web yang digunakan untuk mengumpulkan hal-hal yang menarik berdasarkan kriteria tertentu yang sering dikunjungi di jaman sekarang. Pinterest menggunakan Bahasa Pemrograman Python dari awal mereka membangunnya itulah sebabnya bookmarking (Sebuah metode bagi pengguna internet untuk mengorganisasi, menyimpan, mengelola, dan mecari penanda sumber daya yang tersedia secara online) yang ada di pinterest begitu terstruktur dan mudah untuk diatur.

\section*{\textit{Instagram}}
\par
\textit{Instagram} merupakan sebuah aplikasi berbagi foto dan video secara digital yang digunakan oleh lebih dari 400 juta user yang aktif setiap harinya. Instagram menggunakan bahasa pemrograman python dalam task queuenya atau fitur dimana pada saat yang bersamaan instagram dapat melakukan posting ke beberapa social network lainnya seperti Facebook, Twitter, dll. 

\section*{\textit{Industrial Light and Magic}}
\textit{Indusrial Light and Magic} adalah Studio spesial-efek yang digunakan pada pemutaran efek di film Star Wars. Dalam pembuatan efek ledakan ILM menggunakan Python dikarenakan dapat menghemat waktu dalam pembuatan efek tersebut.   
\par


%include{section/chapter3}
%include{section/chapter4}
%include{section/chapter5}
%include{section/chapter6}
%include{section/chapter7}
%include{section/chapter8}
%include{section/chapter9}
%include{section/chapter10}
%include{section/chapter11}
%include{section/chapter12}
%include{section/chapter13}
%include{section/chapter14}

%now enable appendix numbering format and include any appendices
%appendix
%include{section/appendix1}
%include{section/appendix2}

%next line adds the Bibliography to the contents page
%addcontentsline{toc}{chapter}{Bibliography}
%uncomment next line to change bibliography name to references
%\renewcommand{\bibname}{References}
%bibliography{references}        %use a bibtex bibliography file refs.bib
%bibliographystyle{plain}  %use the plain bibliography style

\end{document}

