\chapter{PERATURAN UMUM}
\section{Sejarah Python}
Python merupakan bahasa pemrograman tingkat tinggi yang dirancang pertama kali oleh  Guido van Rossum pada tahun 1991 dan dikembangkan oleh Python Software Foundation. Python dikembangkan dengan tujuan utama untuk penekanan pada keterbacaan kode, dan sintaksnya memungkinkan programmer untuk mengekspresikan konsep dalam baris kode yang lebih sedikit.
 Guido Van Rossum pertamakali mulai melakukan pekerjaan berbasis aplikasinya pada Desember 1989 di Centrum Wiskunde \& Informatica (CWI) yang terletak di Belanda. Awalnya dimulai sebagai proyek "iseng", karena ia mencari proyek yang menarik untuk membuatnya sibuk selama Natal. Bahasa pemrograman python mulanya berasal dari bahasa pemrograman ABC, Guido van Rossum mengambil sintaks ABC, beberapa fitur yang bagus, dan juga menyelesaikan beberapa "bug" yang ada pada bahasa pemrograman ABC sehingga terlahirlah bahasa scripting yang baik yang telah menghapus semua kekurangan. \\
\par 
Nama python sendiri diambil oleh Guido van Rossum saat dia melihat sebuah acara TV BBC - 'Monty Python’s Flying Circus', karena ia adalah penggemar berat acara TV dan juga ia menginginkan nama pendek, unik dan sedikit misterius untuk penemuannya maka terlahirlah nama Python. 
Bahasa pemrograman python memilki 2 versi yaitu versi 2 dan versi 3, berikut adalah perbedaan kedua versi tersebut.

\begin{enumerate}
\item \textbf{Libraries}
\par 
Banyak developer yang menbuat libraries yang khusus digunakan untuk python 3, libraries untuk python 2 sudah tidak cocok lagi untuk masa mendatang.
\item \textbf{Text String}
\par 
Python 3 memiliki settingan default berupa Unicode, sedangkan python 2 memiliki settingann default berupa ASCII.
\item \textbf{Pembagian}
\par 
Dalam melakukan pembagian, python 2 akan membulatkan hasil bagi jika hasil bagi itu berupa desimal, sedangkan python 3 tidak dibulatkan
\item \textbf{Print Syntax}
\par 
Untuk python 3 print syntax diganti dengan print () function.
\end{enumerate}

\section{Implementasi dan penggunaan Python di perusahaan dunia}
Beberapa perusahaan di dunia mengimplementasikan bahasa pemrograman pyhton, diantaranya adalah sebagai berikut
\begin{enumerate}
\item \textbf{Instagram}
\par 
Instagram menggunakan bahasa pemrograman pyhton versi 3, karena versi terbaru dari python memiliki runtime yang lebih cepat
\item \textbf{Facebook}
\par 
Python saat ini bertanggung jawab atas beberapa layanan dalam manajemen infrastruktur. Ini termasuk menggunakan TORconfig untuk menangani pengaturan switch jaringan, FBOSS untuk CLI switch whitebox, dan menggunakan Dapper untuk penjadwalan dan pelaksanaan pekerjaan pemeliharaan.
\item \textbf{Spotify}
\par 
menggunakan bahasa python terutama untuk analisis data dan layanan back end. Di bagian backend, ada sejumlah besar layanan yang semuanya berkomunikasi melalui 0MQ, atau ZeroMQ, pustaka dan kerangka kerja jaringan sumber terbuka yang ditulis dalam Python dan C ++ (di antara bahasa lain).
\end{enumerate}
