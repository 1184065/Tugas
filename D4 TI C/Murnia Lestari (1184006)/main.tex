\documentclass{article}
\usepackage[utf8]{inputenc}

\title{Resume Sejarah,Instalasi,Identasi Python}
\author{Murnia Lestari (1184006) }
\date{18 October 2019}

\begin{document}

\maketitle

\section { Sejarah Python}
\usepackage{Python merupakan bahasa pemrograman yang berpatokan dengan tingkat pembacaan kode yang cepat yang dikembangkan oleh Guido van Rossum di tahun 1990 yang terletak pada Mathematisch Centrum (CWI),sebagai  versi terbaru dari pemrograman ABC. Tahun 1995, saat itu Guido pidah lokasi ke CNRI di Virginia amerika dengan terus mengembangkan python . setelah itu pada tahun 2000 guido dan beberapa anggota tim pyton  ke Beopen.com. Saat ini perkembangan python masih terus berlanjut oleh bebrapa kumpulan pemrogram yang di koordinasikan Guido dan python software foundation. Nama python bersal dari kecintaan Guido pada sebuah acara televisi yang seringkali muncul dalam korespondensi antar bebrapa pengguna pada python.}

\section {Perbedaan python 2 dan 3}
\usepackage{Python versi  2 adalah bahasa pemrograman yang berpatokan dengan tingkat pembacaan kode yang cepat dan banyak digunakan oleh developer dan di lingkungan  produksi karena pada aplikasi python 2 dilegkapi dengan fitur programatikal. Sedangkan python 3 adalah bahasa pemrograman dengan tingkat pembacaan kode yang tinggi  sehingga python 3 memiliki banyak fitur yang lebih lengkap  karena python 3 merupakan pengembangan dari python 2.Contoh beberapa perbedaan pada python 2 dan 3 :}
\subsection{Cara membuka  python 2 hanya dengan mengetikkan “python” dan untuk membuka python 3 kita harus mengetikkan klue yang jelas yaitu “python 3”.}
\usepackage{}
 \subsection{ Syntak yang digunakan pada python 2 juga berbeda dengan syntak yang digunakan pada python 3. Contohnya adalah sebagai berikut.
}
 \subsubsection{Pada python 2 print digunakan sebagai statement bukan sebuah fungsi. contohnya Print “Hello word”.}

\subsubsection{Pada python 3 Print digunakan sebagai function.contohnya
Print (“ Hello word”)
}
\subsection{Pembagian integer pada Python 2, semua tipe data angka yang tidak mengandung desimal akan dianggap sebagai integer. Pada Python 3, pembagian pada bilangan integer lebih  dianggap intuitif Contoh :
Pada python 2
A =  3
B = 2
A/B = 1

Pada python 3
A = 3
B = 2
A/B =  1.5
}
\section{Implementasi python dan penggunaan di perusahaan kelas dunia}
\usepackage{Pada saat ini python banyak dipakai oleh lingkungan akademis baik din Indonesia ataupun di belahan duni karena python adalah Bahasa yang mudah untuk dipahami bagi pemula . Selain itu pula  perusahaan dunia juga menggunakan python karena python merupakan Bahasa yang multifungsi. Berikut ini adalah contoh perusahaan kelas dunia yang menggunakan python :}

\begin{enumerate}
    \item 	Google
    \item  	Facebook
    \item  Instagram
    \item	Spotify
   \item	Netflix
    \item Quora

\end{enumerate}
\section{Instalasi}
\subsection {instalasi python}
\usepackage {Langkah –langkah mengisntal python}
\begin{enumerate}
\item Download python di python.org
\item Lalu pilih download,lalu pilih versi python yang ingin kamu install contohnya python versi 3.74
\item Lalu setelah terdownload maka buka filenya dan mulailah penginstalan
\item Setelah itu ceklis add python 3.6 to path
\item Lalu pilih install now dan tunggu hingga selesai
\item Jika kamu ingin mengetahui apakah install python mu berhasi kamu bisa mencoba mengetikkan python di mesin pencarian dan pilih python lalu cobalah ketikkan print (“hello word”)
\end{enumerate}
\subsubsection{Instalasi PIP}
\begin{enumerate}
\item	Install pip pada PIP-Installer.org
\item	Lalu pilih instalasi dan download get.pip.py
\item	Setelah pip di download maka jalankan pip
\item	Tunggu hingga terdapat bacaan successfully
\item	Lalu buka terminal/comman prompt
\item	Ketikkan pip
\item Lalu ketikkan pip install request tunggu hingga install nya sukses
\end{enumerate}
\subsubsection{Cara Setting Environment}
\begin{enumerate}
\item	Cara control panel
\item	Klik system and security
\item	Klik system
\item	Klik advance system setting
\item	Pilih environment setting
\item	Dan pilih pathhext, lalu ketikkan C:\Phyton37/
37 adalah versi python yang kalian punya.

\subsubsection{Cara menjalankan enterpreter melalui command prompt/terminal}
\begin{enumerate}
    \item Buka notepad, lalu ketikkan print (“hello word”)
\item	Lalu save dengan nama extensi filenya hell.py
\item Extensi file harus .py yang artinya python.
\item	Lalu buka command prompt
Ketikkan localdisk tempat kamu menyimpan file contoh D:
\item	Lalu ketikkan cd nama folder tempat kamu menyimpan file condoh cd python
\item	Lalu setalah itu ketikkan dir python
\item	Lalu ketikkan hell.py
\item	Maka file yang akan buat akan di cetak pada command prompt
\end{enumerate}
\subsubsection{Cara menjalankan dan mengupdate}
\begin{enumerate}
\item	Buka anaconda navigator
\item Pilih my trade lalu setelah itu akan terbuka command prompt
\item	Setelah itu ketikkan conda update –n my trade –all
\item	Pilih y
\item Pilih exit untuk ekluar ketika telah berhasil mengupdate.
\end{enumerate}
\subsubsection{Cara menjalankan hello word di spyder}
\begin{enumerate}
\item	Buka anaconda navigator lalu pilih spider dan launch
\item	Lalu ketikkan print (“hello word”)
\item	Lalu run
\end{enumerate}
\subsubsection{Menjalankan scrip otomatis login aplikasi otomatis akademik dnegan libarary selneimum dan inputan user}
\begin{enumerate}
    \item 	Buka command promt lalu ketikkan “pip install selenium”
\item	 Setelah itu download geckdriver.exe sesuai dengan versi  yang kalian inginkan.
\item	 Setalah itu buka spyder di anaconda 
\usepackage{
Form Selenium Import Webdriver
Form Selenium.Webdriver.Firefox.Option Import Options
Opsi=Option()
Opsi=Webdriver.Firefox.Options.Options()
Opsi.Headless = False
Binary = Webdriver.Firefox.Firefox_Binary.Firefoxbinary(“C:\Programfiles(X86)\Mozilla Firefoxfirefox
Cap = Webdriver.Common.Desired_Capabilities.Desiredcapabilities().Firefox
Cap[‘Marionette’] = TRUE
Driver = Webdriver.Firefox()
Driver.Get(“Https:\\Siap.Poltekpos.Ac.Id”)
Driver.Find_Element_By_Name(‘User_Name).Send_Keys(“1184006”)
Driver.Find_Element_By_Name(‘Password’).Send_Keys(“Sariasih54”)
Driver.Find_Element_By_Xpath()
}


\end{enumerate}


\subsection{Cara pemakaian variable explorer}
\usepackage{File yang kamu masukkan akan di tampilkan pada explorer variable maka nama,tipe data dan nilai akan keluar di variable explorer
}
\section{Identasi}
\subsection{	Penjelasan identasi}
\usepackage{Berasal dari kata identation yang artinya adalah menjorok masuk ke dalam yang digunakan untuk memisahkan  mana bagian milik IF,dan mana bagian else. Karena itu python mengharuskan adanya identasi.}
\subsection{Jenis – jenis error  }
\usepackage{Identasi ini bisa terjadi  ketika salah memberi identasi oleh karena itu identasi dijadikan sebagai penanda blok.}
\subsection{	Cara mmebaca error pada identasi}
\begin{enumerate}
    \item Lihat pada script jika terdapat symbol warning itu artinya terjadi error
\item	Lihat pada consol jika tedapat error maka terdapat bacaan yang berwarn amerah
\end{enumerate}

\subsection{Cara menangani error}	
 \usepackage{ Caranya adalah tanda warning atau error pada console kalian lihat , lalu cari kesalahannya serta solusi untuk error tersebut.s} 
 \end{enumerate}
                         
 \end{document}
