\chapter{Komunikasi Perangkat Keras}

Tujuan pembelajaran pada pertemuan kelima antara lain:
\begin{enumerate}
\item
Mengenal komunikasi data serial
\item
Mengerti cara memakai library PySerial
\item
Mengerti cara instalasi driver dan menemukan BaudRate dan Nomor Port
\item
Mengatasi Error yang terjadi akibat pemakaian library csv dan pandas
\item
Try Except
\end{enumerate}
Tugas dengan cara dikumpulkan dengan pull request ke github dengan menggunakan latex pada repo yang dibuat oleh asisten IRC. Kode program dipisah dalam folder src NPM.py yang berisi praktek dari masing-masing tugas file terpisah sesuai nomor yang kemudian dipanggil menggunakan input listing ke dalam file latex penjelasan atau nomor pengerjaan. Masing masing soal bernilai 5 dengan total nilai 100. Gunakan bahasa yang baku dan bebas plagiat dengan dibuktikan hasil scan plagiarisme. Serta hasil scrinsut dari komputer sendiri, dan kode hasil sendiri. Pengerjaan menggunakan latex dan harus menyertakan file pdf hasil compile pdflatex, jika tidak diskon 50\%.


\section{Pemahaman Teori}
Kerjakan soal berikut ini, masing masing bernilai 5. Untuk hari pertama.
Praktek teori penunjang yang dikerjakan dengan deadline rabu jam 4 pagi:
\begin{enumerate}
\item
Apa itu fungsi device manager di windows dan folder /dev di linux
\item
Jelaskan langkah-langkah instalasi driver dari arduino
\item
Jelaskan bagaimana cara membaca baudrate dan port dari komputer yang sudah terinstall driver
\item
Jelaskan sejarah library pyserial
\item
Jelaskan fungsi-fungsi apa saja yang dipakai dari library pyserial
\item
Jelaskan kenapa butuh perulangan dalam tidak butuh perulangan dalam membaca serial
\item
Jelaskan bagaimana cara membuat fungsi yang mengunakan pyserial
\end{enumerate}

\section{Ketrampilan Pemrograman}
Kerjakan soal berikut ini, masing masing bernilai 10 untuk hari kamis jam 4 pagi. Soalnya adalah:

\begin{enumerate}
\item
Buatlah fungsi (file terpisah/library dengan nama NPM\_realtime.py) untuk mendapatkan data langsung dari arduino
\item
Buatlah fungsi (file terpisah/library dengan nama NPM\_save.py) untuk mendapatkan data langsung dari arduino dengan looping
\item
Buatlah fungsi (file terpisah/library dengan nama NPM\_realtime.py) untuk mendapatkan data dari arduino dan langsung ditulis kedalam file csv
\item
Buatlah fungsi (file terpisah/library dengan nama NPM\_csv.py) untuk membaca file csv hasil arduino dan mengembalikan ke fungsi
\end{enumerate}




\section{Ketrampilan Penanganan Error}
Kerjakan soal berikut ini, masing masing bernilai 5(hari kedua). Bagian Penanganan error dari script python.
\begin{enumerate}
\item
Tuliskan peringatan error yang didapat dari mengerjakan praktek ketiga ini, dan jelaskan cara penanganan error tersebut.
dan Buatlah satu fungsi yang menggunakan gunakan try except untuk menanggulangi error tersebut.
\end{enumerate}



\section{Presentasi Tugas}
Pada pertemuan ini, diadakan dua penilaiain yaitu penilaian untuk tugas mingguan seperti sebelumnya dengan nilai maksimal 100. Kemudian dalam satu minggu kedepan maksimal sebelum waktu mata kuliah pemrograman 3. Ada presentasi kematerian dengan nilai presentasi yang terpisah masing-masing 100. Jadi ada tiga komponen penilaiain pada pertemuan ini yaitu :
\begin{enumerate}
	\item tugas minggu hari ini dan besok (maks 100). pada chapter ini
	\item presentasi pyserial (maks 100). Mempraktekkan kode python dan menjelaskan cara kerjanya.
\end{enumerate}
Waktu presentasi pada jam kerja di IRC. Kriteria penilaian presentasi sangat sederhana, presenter akan ditanyai 20(10 pertanyaan program, 10 pertanyaan teori) pertanyaan tentang pemahamannya menggunakan python untuk kecerdasan buatan. jika presenter tidak bisa menjawab satu pertanyaan asisten maka nilai nol. Jika semua pertanyaan bisa dijawab maka nilai 100. Presentasi bisa diulang apabila gagal, sampai bisa mendapatkan nilai 100 dalam waktu satu minggu kedepan.




