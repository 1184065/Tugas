\documentclass{article}
\usepackage[utf8]{inputenc}

\title{Resume Python}
\author{nurulkamila1899 }
\date{October 2019}

\begin{document}

\maketitle

\section{Sejarah Python }
\usepackage{Python adalah Bahasa pemrograman interpretative multiguna dengan filosofi perancangan yang berfokus pada tingkat keterbacaan kode. Python dikembangkan oleh Guido van Rossum pada tahun 1990 di Stichting Mathematicsch Centrum (CWI), Amsterdam sebagai kelanjutan dari Bahasa pemrograman ABC. Saat 1995, Guido pindah ke CNRI di Virginia Amerika sambil terus melanjutkan pengembangan Python. Pada  2000, Guido dan para pengembang inti Python pindah ke BeOpen.com. Setelah itu, Guido dan beberapa anggota tim PythonLabs pindah ke DigitalCreations.}\\

\usepackage{Guido dan Python Software Foundation yang merupakan sebuah organisasi non-profit yang dibentuk sebagai pemegang hak cipta intelektual Python sejak versi 2.1  terus melakukan  pengembangan Python dan dengan demikian mencegah Python dimiliki oleh perusahaan komersial. }\\

\usepackage{Sebutan  Python dipilih oleh Guido sebagai nama bahasa ciptaannya karena kecintaan Guido pada acara televisi Monty Python’s Flying Circus. Dengan demikian seringkali ungkapan-ungkapan khas dari acara tersebut seringkali muncul dalam korespondensi antar pengguna Python.}

/
\section{Perbedaan Python 2 dan 3}
\subsection{	Python 2}
\usepackage{Dipublikasikan pada akhir tahun 2000. Python 2 dinilai lebih transparan dan inklusif untuk pengembangan software ketimbang versi sebelumnya. Hal ini didukung dengan adanya PEP-Python Enhancement Proposal, sebuah spesifikasi teknis yang menjadi tuntunan informasi untuk penggunaanya dan menggambarkan fitur baru pada Python itu sendiri.}\\

\usepackage{Python 2 dilengkapi dengan berbagai fitur programatikal seperti cycle-detecting garbage collector untuk mengotomasi manajemen memori, peningkatan dukungan untuk  Unicode, list comprehension untuk membuat sebuah list berdasarkan list yang sudah ada, unifikasi pada tipe data python dan class ke satu hirarki terjadi pada rilis Python 2.2}

\subsection{	Python 3}
\usepackage{Dirilis pada akhir tahun 2008, dengan focus untuk melakukan perapian pada codebase  dan menghapuskan duplikasi (redundancy). Awalnya, Python 3 mengalami hambatan pada pengadopsiannya. Itu akibat dari tidak adanya backwards compatibility dengan Python 2. Tambahannya banyak sekali library yang hanya tersedia untuk Python 2.}

\section{Implementasi Python}
\usepackage{CPython adalah implementasi Python asli. CPython  kebetulan diimplementasikan di C. CPython mengkompilasi kode python menjadi bytecode (transparan) dan menafsirkan bytecode dalam lingkaran evaluasi.}\\

\usepackage{Jython, IronPython, dan PyPy adalah implementasi lainnya dari Bahasa pemrograman Python. Jython diimplementasikan di Java, C# dan RPython (subset dari Python), di implementasikan pada lingkungan masing-masing. Jython mengkompilasi kode Python ke Java bytecode, jadi kode Python dapat berjalan di JVM. IronPython memungkinkan kita menjalankan Python di Microsoft CLR. Dan PyPy yang diimplementasikan di (subset dari) Python memungkinkan kita menjalankan kode Python lebih cepat dari CPython menggunakan JIT Compiler. Ada sebuah proyek yang menerjemahkan kode Python ke C dan itu disebut Cython.}

\section{Beberapa perusahaan dunia yang berhasil memperbaiki dan mengembangkan aplikasi menggunakan Python:}
\subsection{	Spotify}
\usepackage{Penyedia layanan streaming music Spotify memanfaatkan Python untuk analisis data dan backend.}
\subsection{	Google}
\usepackage{Dari awal berdiri, Google sudah menggunakan Python, bahkan Python merupakan salah satu Bahasa pemrograman yang penting bagi Google.}
\subsection{	Cahaya dan Sihir Industri}
\usepackage{Menggunakan Python, ILM dapat dengan mudah membungkus perangkat lunak dan meningkatkan aplikasi grafis mereka.}
\subsection{	Netflix}
\usepackage{Salah satu penggunaan utama Python di aplikasi Netflix adalah pada Central Alert Gateway.}
\subsection{	Instagram}
\usepackage{Bagi para pengembang aplikasi di Instagram Python sangat ramah pengguna, mudah dan rapi.}
\end{document}
