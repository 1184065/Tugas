\chapter{Pengelolaan File CSV}
\section{Teori}
\subsection{Sejarah dan Contoh}
File comma-separated values (CSV) adalah file teks yang dibatasi menggunakan koma untuk memisahkan nilai. File CSV menyimpan data tabular/table (angka dan teks) dalam teks biasa. Setiap baris file adalah catatan data. Setiap catatan terdiri dari satu atau lebih bidang. Penggunaan koma sebagai pemisah bidang adalah sumber nama untuk format file ini.\\
\textbf{Contoh:}
\begin{lstlisting}
No, Nama, Jurusan
1, Rayhan, D4 Teknik Informatika
2, Dadang, D4 Teknik Informatika
\end{lstlisting}

\subsection{Aplikasi Pembuat File CSV}
Berikut merupakan aplikasi-aplikasi yang dapat membuat file csv.
\begin{itemize}
\item Google sheet
\item Ms.Excel
\item Notepad
\item Numbers (MacOS)
\item Wordpad
\item Dll.
\end{itemize}

\subsection{Cara Membaca dan Menulis File CSV di Excel atau Spreadsheet}
\begin{itemize}
\item Jalankan Program Excel atau Spreadsheet Sejenis.
\item Input Data Pada Baris dan Kolom, Contoh:
\item Lalu pilih save as, dan save file dengan format .csv
\end{itemize}

\subsection{Library Pada CSV}
\lstinputlisting[language=Python]{src/reader.py}
\lstinputlisting[language=Python]{src/dictreader.py}
\lstinputlisting[language=Python]{src/write.py}
\lstinputlisting[language=Python]{src/dictwrite.py}


\subsection{Library Pada Pandas}
\lstinputlisting[language=Python]{src/read_csv.py}
\lstinputlisting[language=Python]{src/to_csv.py}