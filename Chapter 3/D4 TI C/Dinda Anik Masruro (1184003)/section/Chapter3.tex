\section{Fungsi}
	\subsection{pemahaman teori}
		\begin{enumerate}
			\item fungsi\\
			Fungsi adalah blok kode yang akan dieksekusi ketika dipanggil dalam se-
            buah program.\\
			
			\item parameter adalah inputan sebuah fungsi yang bertujuan untuk menyimpan sebuah nilai.
			
			\item return berfungsi untuk mengembalikan nilai yang telah di proses dalam suatu fungsi dan mengakhiri sebuah fungsi
			\begin{lstlisting}[language=Python]
			def fungsi(x,y):
					z=x+y
					return z
			\end{lstlisting}
			
			\item item paket adalah sebuah direktory dengan file python dan file dengan nama \_init\_.py. jadi sebuah direktori didalam sebuah python dengan nama \_init\_.py, akan dianggap sebagai paket oleh python.untuk memanggil sebuah paket atau library adalah dengan cara \textit{import} nama paket atau library tersebut lalu paket atau library tersebut dapat di gunakan.
			\begin{lstlisting}[language=Python]
			from kampus import mahasiswa
			\end{lstlisting}
			
			\item class adalah sebuah blueprint dari sebuah objek yang akan di bangun.
			\begin{lstlisting}[language=Python]
class World:
    def __init__(self,World):
        self.World = World
    def heloWorld(self):
        print("Helo",World)
			\end{lstlisting}
			
			\item objek memiliki variable dan kode yang saling terhubung. objek di buat dengan class.
			\begin{lstlisting}[language=Python]
#import kelas terlebih dahulu
import kelas3lib
#membuat object
cobakelas=kelas3lib.Kelas3ngitung(npm) 
hasilkelas=cobakelas.npm1()
			\end{lstlisting}
			
			\item attribut adalah sebuah tempat tampungan sebuah data atau perintah yang berhubungan dengan attribut tersebut
			\begin{lstlisting}[language=Python]
class Kelas3ngitung:
	#pendefinisian attribute
    def __init__(self,World):
        self.World = World
			\end{lstlisting}
			
			\item method adalah sebuah fungsi dalam class.
			\begin{lstlisting}[language=Python]
class world:
    def __init__(self,world):
        self.world = world
    #Pembuatan method pada class
    def world(self):
       	print("hello",world,",apa kabar ?")
			\end{lstlisting}
			
			\item contoh membuat sebuah library, contoh disini kita membuat pada folder libra :
			\begin{lstlisting}{language=Python}
def hello():
    print("Hello world")
			\end{lstlisting}

            \newpage
			\item contoh jika kita ingin memanggil fungsi dari library pada main program kita harus terlebih dahulu import :
			\begin{lstlisting}{language=Python}
#import library yang telah dibuat
import library
#pemanggilan fungsi pada library
library.hello()
			\end{lstlisting}
			
			\item pemakaian package from kalkulator import penambahan
			\begin{lstlisting}{language=Python}
from kalkulator import penambahan
			\end{lstlisting}
kode diatas berarti program memanggil sebuah package terlebih dahulu baru menambahkan source code penambahan, kode diatas dapat dibaca seperti ini "import penambahan dari folder kalkulator"

			\item pemanggilan library dalam sebuah folder\\
			untuk mengakses sebuah library dalam sebuah folder kita perlu menuliskan foldernya terlebih dahulu lalu mengimport nama librarynya, contoh :
			\begin{lstlisting}{language=Python}
from mahasiswa import kampus
			\end{lstlisting}
artinya dalam package mahasiswa akan memakai library kampus

			\item pemanggilan class dalam sebuah folder\\
			untuk mengakses sebuah class dalam sebuah folder kita perlu menuliskan foldernya terlebih dahulu lalu mengimport nama class nya, contoh :
			\begin{lstlisting}{language=Python}
from mahasiswa import kampus
			\end{lstlisting}
artinya dalam package mahasiswa kita akan memakai class kampus
			
			\end{enumerate}

    \newpage			
	\subsection{Ketrampilan Pemrograman}
			\begin{enumerate}
				\item \lstinputlisting[language=Python]{src/npm1.py}
				\item \lstinputlisting[language=Python]{src/npm2.py}
				\item \lstinputlisting[language=Python]{src/npm3.py}
				\item \lstinputlisting[language=Python]{src/npm4.py}
				\item \lstinputlisting[language=Python]{src/npm5.py}
				\item \lstinputlisting[language=Python]{src/npm6.py}
				\item \lstinputlisting[language=Python]{src/npm7.py}
				\item \lstinputlisting[language=Python]{src/npm8.py}
				\item \lstinputlisting[language=Python]{src/npm9.py}
				\item \lstinputlisting[language=Python]{src/npm10.py}
				\item \lstinputlisting[language=Python]{src/lib3.py}
				\item \lstinputlisting[language=Python]{src/kelas3lib.py}
				\item \lstinputlisting[language=Python]{src/main.py}
	
\end{enumerate}