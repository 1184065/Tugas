\documentclass{article}
\usepackage[utf8]{inputenc}
\usepackage{listings}

\usepackage{color}
 
\definecolor{codegreen}{rgb}{0,0.6,0}
\definecolor{codegray}{rgb}{0.5,0.5,0.5}
\definecolor{codepurple}{rgb}{0.58,0,0.82}
\definecolor{backcolour}{rgb}{0.95,0.95,0.92}
 
\lstdefinestyle{mystyle}{
    backgroundcolor=\color{backcolour},   
    commentstyle=\color{codegreen},
    keywordstyle=\color{magenta},
    numberstyle=\tiny\color{codegray},
    stringstyle=\color{codepurple},
    basicstyle=\footnotesize,
    breakatwhitespace=false,         
    breaklines=true,                 
    captionpos=b,                    
    keepspaces=true,                 
    numbers=left,                    
    numbersep=5pt,                  
    showspaces=false,                
    showstringspaces=false,
    showtabs=false,                  
    tabsize=2,
    language=python
}
 
\lstset{style=mystyle}



\title{Tugas Pemrograman Section 3}
\author{Ahmad Agung Tawkkal\\
        1184015\\
        D4 TI 1B}
\date{October 2019}

\begin{document}

\maketitle

\newpage

\section{Teori}
    
    \subsection{Fungsi}
        \paragraph{}Fungsi adalah suatu balok program yang terdiri dari nama fungsi, input variable dan variable kembalian. Nama funsi diawali dengan \textit{def} dan setelahnya tanda titik dua.Nama bisa sama dengan isi berbeda jika menggunakan huruf besar dan kecil atau sering disebut dengan \textit{case sensitive}. Input variable bisa lebih dari satu dengan pemisahan tanda koma. Variable kembalian pasti satu, bebas apakah itu jenis \textit{string, integer, list} atau \textit{dictionary}. Contohnya sebagai berikut:
            \lstinputlisting[language=Python]{src/fungsi.py}
    \subsection{Library}    
        \paragraph{}Library digunakan untuk memanggil funsi pada file python yang berbeda, library bertujuan untuk memanggil fungsi agar dapat memisahkan program di setiap fungsi.
        \paragraph{}Pada file dibawah terdapat ada dua fungsi yaitu \textit{salam} dan \textit{nama lengkap} 
            \lstinputlisting[language=Python]{src/fungsi1.py}
        
        \paragraph{}Kemudaian untuk memanggil fungsinya anda tinggal mengimport.
            \lstinputlisting[language=Python]{src/library.py}
    
    \subsection{Kelas, Objek, Atribut, dan Method}
        \subsubsection{Kelas}
            \paragraph{}Kelas menyediakan sarana untuk menggabungkan data dan fungsionalitas bersama. Membuat kelas baru menciptakan tipe objek baru, memungkinkan instance baru dari tipe itu dibuat. Setiap instance kelas dapat memiliki atribut yang melekat padanya untuk mempertahankan negaranya. Instance kelas juga dapat memiliki metode (ditentukan oleh kelasnya) untuk memodifikasi kondisinya.
        \subsubsection{Objek}
            \paragraph{}Pemrograman berorientasi objek atau dalam bahasa inggris disebut \textit{Object Oriented Programming} (OOP) adalah paradigma atau teknik pemrograman di mana semua hal dalam program dimodelkan seperti objek dalam dunia nyata. Objek di dunia nyata memiliki ciri atau attribut dan juga aksi atau kelakuan (\textit{behaviour}).
        \subsubsection{Atribut}
            \paragraph{}Atribut adalah nilai data yang terdapat pada suatu object yang berasal dari class. Attributes merepresentasikan karakteristik dari suatu object.
        \subsubsection{Method}
            \paragraph{}Method adalah fungsi atau prosedur yang dibuat oleh seorang programmer didalam suatu Class. Method dapat dibagi menjadi fungsi dan prosedur. Fungsi adalah bagian atau sub dari program yang mempunyai algoritma tertentu dalam menyelesaikan suatu masalah dengan mengembalikan hasil. Prosedur adalah bagian atau sub dari program yang mempunyai algoritma tertentu dalam menyelesaikan suatu masalah tanpa mengembalikan suatu nilai hasil. 
                \lstinputlisting[language=Python]{src/kelas.py}
                
    \subsection{Penggunaan Library}
        \paragraph{}Contoh file yang akan dipanggil dengan from :
            \lstinputlisting[language=Python]{src/fungsi1.py}
        \paragraph{}Contoh pemanggilan library:
            \lstinputlisting[language=Python]{src/library1.py}
            
    \subsection{Import Kalkulator}
        \lstinputlisting[language=Python]{src/kalkulator.py}
        \lstinputlisting[language=Python]{src/main.py}
        \paragraph{}Output = 5
    
    \subsection{Pemanggilan library dalam sebuah folder}
        \paragraph{}Untuk mengakses sebuah library dalam folder, terlebidahulu foldernya kita tulis(src) kemudian mengimport nama librarynya(soal1).
        
        Contoh: from src import libra
    
    \subsection{Pemanggilan class dalam sebuah folder}
        \paragraph{}Untuk mengakses sebuah class dalam sebuah folder, terlebihdahulu menuliskan foldernya, kemudian mengimport nama class-nya.
        
        contoh: from src import Buah
    
\section{Keterampilan Pemrograman}

    \section*{Soal 1}
        \lstinputlisting[language=Python]{soal/soal1.py}
        
    \section*{Soal 2}
        \lstinputlisting[language=Python]{soal/soal2.py}
        
    \section*{Soal 3}
        \lstinputlisting[language=Python]{soal/soal3.py}
    
    \section*{Soal 4}
        \lstinputlisting[language=Python]{soal/soal4.py}
    
    \section*{Soal 5}
        \lstinputlisting[language=Python]{soal/soal5.py}
        
    \section*{Soal 6}
        \lstinputlisting[language=Python]{soal/soal6.py}
        
    \section*{Soal 7}
        \lstinputlisting[language=Python]{soal/soal7.py}
        
    \section*{Soal 8}
        \lstinputlisting[language=Python]{soal/soal8.py}
    
    \section*{Soal 9}
        \lstinputlisting[language=Python]{soal/soal9.py}
    
    \section*{Soal 10}
        \lstinputlisting[language=Python]{soal/soal10.py}
    
    \section*{Soal 11}
        \lstinputlisting[language=Python]{soal/lib3.py}
    
    \section*{Soal 12}
        \lstinputlisting[language=Python]{soal/kelas3lib.py}
        \lstinputlisting[language=Python]{soal/main.py}
    
\end{document}
