\chapter{Fungsi dan Kelas}
\section{Fungsi}
\subsection{Fungsi}
\par 
Fungsi merupakan suatu blok kode yang berfungsi untuk menampung suatu bari program yang nantinya dapat dieksekusi dengan cara memanggil fungsi tersebut.

\subsection{Parameter}
\par
Parameter adalah sebuah variabel yang dapat menampung suatu nilai yang nantinya dijalan kan pada sebuah fungsi, contohnya.

\begin{lstlisting}[language=Python]

def npm(npm):
	print("hai")
	
\end{lstlisting}

\subsection{Mengembalikan Nilai(\textit{Return}}
\par
Return merupakan sebuah fungsi yang digunakan untuk menampilkan output dari fungsi yang sebelumnya telah dibuat

\begin{lstlisting}[language=Python]
def keliling(kotak):
	keliling = p * l
	return keliling 
\end{lstlisting}

\section{Package}
\subsection{Package}
Package adalah sebuah wadah untuk menyimpan sekumpulan file-file modul.
Cara memanggil sebuah package adalah sebagai berikut

\begin{lstlisting}[language=Python]
form mahasiswa input npm
\end{lstlisting}

\section{Class}
\subsection{Class}
Class merupakan sebuah prototipe/blueprint dari sebuah objek. contohnya class ini akan diberi nama \textbf{satu.py}

\begin{lstlisting}[language=Python]
class Mahasiswa:
	def __init__(self,npm):
		self.npm = npm
	def mhs(self,npm):
		print(npm)
\end{lstlisting}


\subsection{Objek}
Objek merupakan hasil yang telah terdefinisikan dari sebuah class.

\begin{lstlisting}[language=Python]
import satu

test=satu.npm(npm)
\end{lstlisting}

\subsection{Atribut}
Atribut merupakan variabel yang dimiliki suatu class

\begin{lstlisting}[language=Python]
class Mahasiswa:
	def __init__(self,npm):
		self.npm = npm
\end{lstlisting}

\subsection{Method}
Method merupakan kumpulan fungsi-fungsi pada sebuah class

\begin{lstlisting}[language=Python]
class Mahasiswa:
	def __init__(self,npm):
		self.npm = npm
	def mhs(self,npm):
		print(npm)
\end{lstlisting}


\begin{lstlisting}[language=Python]
import satu

test=satu.npm(npm)
\end{lstlisting}

\section{Package}
\subsection{Penggunaan Package}
Buat suatu library terlebih dahulu

\begin{lstlisting}[language=Python]
class Mahasiswa:
	def __init__(self,npm):
		self.npm = npm
	def mhs(self,npm):
		print(npm)
\end{lstlisting}

Import library yang tadi sudah dibuat, dan panggil fungsi yang dibutuhkan
\begin{lstlisting}[language=Python]
import satu

test=satu.npm(npm)
\end{lstlisting}

\section{Import}
\subsection{from kalkulator import penambahan}
\begin{lstlisting}[language=Python]
from kalkulator import penambahan
\end{lstlisting}

Kode tersebut memiliki arti memanggil package kalkulator dan mengimport fungsi penambahan. Contoh code lainnya adalah sebagai berikut.
\begin{lstlisting}[language=Python]
from mahasiswa import npm
\end{lstlisting}

\section{Library}
\subsection{Pemanggilan library dalam folder}
Untuk memanggil sebuah library, pertama kita harus memanggil foldernya terlebih dahulu baru memanggil library yang diiinginkan.
\begin{lstlisting}[language=Python]
from mahasiswa import npm
\end{lstlisting}

\subsection{Pemanggilan class dalam folder}
Untuk memanggil sebuah class, pertama kita harus memanggil foldernya terlebih dahulu baru memanggil library yang diiinginkan.
