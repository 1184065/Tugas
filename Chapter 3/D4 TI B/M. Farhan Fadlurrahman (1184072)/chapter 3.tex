\documentclass[12pt]{article}

\usepackage[T1]{fontenc}
\usepackage{xcolor}
\usepackage{listings}

\begin{document}

\title{Chapter 3 \\ fungsi dan kelas}
\author{M. Farhan F. (1184072)}
\date{}
\maketitle

\section{pemahaman Teori}
	\begin{enumerate}
		\item 
			\begin{itemize}
				\item fungsion adalah blok kode teroganisir yang digunakan untuk melakukan sebuah tindakan atau action dan bisa di gunakan kembali. fungsion diawali dengan \textit{def} kemudain nama fungsion lalu parameter kemudain titik dua dan di akhiri \textit{return} untuk mengakhiri funsion.
				\lstinputlisting[language=Python]{src/n1.py}			
				
				\item parameter adalah untuk menyimpan nilai 
				\lstinputlisting[language=Python]{src/n2.py}
				
				\item inputan kembalian adalah untuk mengembalikan nilai yang telah di proses dalam suatu fungsi dan juga bisa untuk mengakhiri sebuah fungsi				
				\lstinputlisting[language=Python]{src/n3.py}				
				
			\end{itemize}
			
		\item 
			\begin{itemize}
			\item paket adalah sebuah direktory dengan file python dan file dengan nama \_init\_.py. jadi sebuah direktori didalam sebuah python dengan nama \_init\_.py, akan dianggap sebagai paket oleh python.			 
			\item untuk memanggil sebuah paket atau library adalah dengan cara \textit{import} nama paket atau library tersebut lalu paket atau library tersebut dapat di gunakan.
			\lstinputlisting[language=Python]{src/n4.py}			 			
			 		
			\end{itemize}
			
		\item 
			\begin{itemize}
				\item kelas adalah sebuah blueprint dari sebuah objek yang akan di bangun
				\lstinputlisting[language=Python]{src/n5.py}			
				\item objek memiliki variable dan kode yang saling terhubung. objek di buat dengan class.
				\lstinputlisting[language=Python]{src/n6.py}
				\item adalah sebuah tempat tampungan sebuah data atau perintah yang berhubungan dengan attribut tersebut
				\lstinputlisting[language=Python]{src/n7.py}
				\item method adalah sebuah fungsi dalam class.
				\lstinputlisting[language=Python]{src/n5.py}
				\lstinputlisting[language=Python]{src/n6.py}
			\end{itemize}			
			
		\item cara pemanggilan kelas library adalah hampir sama dengan pemanggilan library yaitu dengan menggunakan \textit{import} dahulu tetapi untuk penggunaan klass library menggunakan penambahan variable yang menjadi objek dari kelas.
		\lstinputlisting[language=Python]{src/n9.py}
		
		\item program memanggil sebuah package terlebih dahulu baru menambahkan source code penambahan. contoh lain :
		\lstinputlisting[language=Python]{src/n10.py}
		
		\item untuk mengakses sebuah library dalam sebuah folder lain, perlu menuliskan nama folder kemudian lalu mengimport nama librarynya :
		\lstinputlisting[language=Python]{src/n11.py}
		
		\item sama seperti mengimport library yaitu deangan menuliskan nama folder kemudain lalu mengimport nama classnya :
		\lstinputlisting[language=Python]{src/n12.py}
		
			
		
	\end{enumerate}

\section{Keterampilan pemograman}
	\begin{enumerate}
		\item[Soal No 1] \lstinputlisting[language=Python]{src/n13.py}

		\item[Soal No 2] \lstinputlisting[language=Python]{src/n14.py}
		
		\item[Soal No 3] \lstinputlisting[language=Python]{src/n15.py}
		
		\item[Soal No 4] \lstinputlisting[language=Python]{src/n16.py}
		
		\item[Soal No 5] \lstinputlisting[language=Python]{src/n17.py}
		
		\item[Soal No 6] \lstinputlisting[language=Python]{src/n18.py}
		
		\item[Soal No 7] \lstinputlisting[language=Python]{src/n19.py}
		
		\item[Soal No 8] \lstinputlisting[language=Python]{src/n20.py}
		
		\item[Soal No 9] \lstinputlisting[language=Python]{src/n21.py}
		
		\item[Soal No 10] \lstinputlisting[language=Python]{src/n22.py}
		
		\item[Soal No 11] \lstinputlisting[language=Python]{src/n23.py}
		
		\item[Soal No 12] \lstinputlisting[language=Python]{src/kelas3lib.py} \lstinputlisting[language=Python]{src/main.py}
		
	\end{enumerate}
	
	
\section{Keterampilan penanganan error}
		\begin{enumerate}
			\item File "C:/Users/SAM/Desktop/M. Farhan Fadlurrahman (1184072)/src/n23.py", line 149, in <module>
    prima()\\

TypeError: prima() missing 1 required positional argument: 'npm'\\\\
		error disini terjadi karena dalam memanggil sebauh method tidak parameter dalam pemanggilan tersebut tudak di tulis maka dari itu parameter harus di tulis sama dengan parameter method saat memanggil.
		\item[Soal No 11] \lstinputlisting[language=Python]{src/error.py}
		


		\end{enumerate}

\end{document}