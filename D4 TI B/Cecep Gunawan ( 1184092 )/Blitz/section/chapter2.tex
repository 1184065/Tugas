\chapter{LANDASAN TEORI}
\section{Teori Umum}
\subsection{\textit{e-ticket}}
\par
E-Ticket adalah singkatan dari electronic ticket, atau dalam bahasa Indonesia diartikan tiket elektronik. Sehingga e-ticket adalah tiket yang wujudnya berbentuk elektronik \cite{masuara2015rancang}.
\subsection{\textit{Website}}
\par
\textit{Website} adalah kumpulan dari halaman-halaman situs, yang biasanya terangkum dalam sebuah \textit{domain} atau \textit{subdomain}, yang tempatnya berada di dalam \textit{World Wide Web} (WWW) di \textit{internet} \cite{trimarsiah2017analisis}

\subsection{\textit{Analisis}}
\par
Analisis adalah suatu upaya penyelidikan untuk melihat, mengamati,
mengetahui, menemukan, memahami, menelaah, mengklasifikasi, dan mendalami serta menginterpretasikan fenomena yang ada \cite{astutik2019analisis}


\subsection{\textit{PHP}}
\par
\textit{PHP} merupakan bahasa berbentuk \textit{script} yang ditempatkan didalam \textit{server} baru kemudian diproses. Kemudian hasil pemrosesan dikirim kepada \textit{web browser client}. Bahasa pemrograman ini dirancang khusus untuk membentuk \textit{web} dinamis. Artinya, pemrograman \textit{PHP} dapat membentuk suatu tampilan berdasarkan permintaan terkini, misalnya halaman yang menampilkan daftar tamu. Halaman tersebut akan selalu mengalami perubahan mengikuti jumlah data tamu yang telah mengisi buku tamu \cite{rubiati2018aplikasi}

\subsection{\textit{MySQL}}
\par
\textit{MySQL} adalah \textit{Relational Database Management System} (RDBMS) yang diditribusikan secara gratis dibawah licensi GPL \textit{(General Public License)}. \textit{MySQL} sebenarnya merupakan turunan salah satu konsep utama dalam database sejak lama yaitu \textit{SQL (Structured Query Language)}.\textit{SQL} adalah sebuah konsep pengoperasian database terutama untuk pemilihan/seleksi dan pemasukan data yang memungkinkan pengoperasian data dikerjakan dengan mudah dan secara otomatis. Keandalan suatu sistem \textit{database} dapat diketahui dari cara kerja \textit{optimizer}-nya dalam melakukan proses perintah-perintah \textit{SQL}, yang dibuat oleh \textit{user} maupun program-program aplikasinya. Sebagai \textit{database server}, \textit{MySQL} dapat dikatakan lebuh unggul dibandingkan \textit{database server} lainnya dalam \textit{query} data. Hal ini terbukti untuk \textit{query} yang dilakukan oleh \textit{single user}, kecepatan \textit{query My SQL} dapat sepuluh kali lebih cepat dari PostgreSQL dan lima kali lebih cepat dibandingkan \textit{Interbase} \cite{santi2015implementasi}

\subsection{\textit{Flowchart}}
\par
\textit{Flowchart} merupakan bagan \textit{(chart)} yang menunjukan alir atau arus \textit{(flow)} di dalam program atau prosedur \textit{system} secara logika. \textit{Flowchart} (bagan alir) merupakan gambaran dalam bentuk diagram alir dari \textit{algoritma-algoritma} dalam suatu program, yang menyatakan arah alur program tersebut.
\cite{solikin2018implementasi}

\subsection{\textit{Data Flow Diagram}}
\textit{Data Flow Diagram} (DFD) adalah salah satu yang tertua
alat terstruktur tersedia untuk mendukung analisis sistem dan
Desain. Nantinya DFD akan digunakan sebagai alat untuk menganalisa cara kerja yang ada di sistem tersesebut
 \cite{sauter2015making}
\par

\subsection{\textit{API}}
\par 
\textit{API} atau  \textit{Application  Programming Interface} bukan  hanya satu set \textit{class} dan \textit{method} atau fungsi dan \textit{signature} yang  sederhana.  Akan  tetapi  \textit{API},  yang  bertujuan utama untuk mengatasi \textit{“clueless”} dalam membangun \textit{software}  yang    berukuran besar, berawal dari sesuatu yang sederhana sampai ke yang kompleks dan merupakan perilaku komponen yang sulit dipahami. Secara   sederhana dapat dipahami dengan membayangkan  kekacauan  yang akan  timbul  bila  mengubah \textit{database}  atau  skema \textit{XML}. Perubahan  ini  dapat   dipermudah dengan bantuan \textit{API} \cite{rosdania2016sistem}.

\subsection{Google}
\par 
Google adalah suatu mesin pencari yang sangat trend di zaman sekarang ini, google sangat banyak digunakan oleh manusia untuk membantu mencari informasi, baik itu dalam proses belajar mengajar, mengetahui berita, lowongan pekerjaan dan lain sebagainya. Dimana penggunaan google sangat mudah dan kebanyakan \textit{user} sangat suka menggunakan google. Seiring berkembangnya zaman google semakin canggih dan semakin banyak pengguna yang menggunakan google. Sehingga perpustakaan kurang diminati dalam pencarian informasi \cite{lestari2016klasifikasi}.
