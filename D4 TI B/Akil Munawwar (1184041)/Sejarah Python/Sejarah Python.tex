\documentclass{article}
\begin{document}
\title{Resume Sejarah Python}
\author{Akil Munawwar \\ D4 TI 2B \\ 1184041}
\maketitle

\part{Python}
\section{Sejarah Python}
Python diciptakan oleh Guido van Rossum pertama kali di \textit{Centrum Wiskunde} dan \textit{Informatica} (CWI) di Belanda pada awal tahun 1990-an. Bahasa Python itu sendiri terinspirasi dari bahasa pemrograman ABC. Semua versi python yang dirilis bersifat open source dan menggunakan lisensi GFL-compatible.
\section{Perbedaan Python 2 dan 3}
Perbedaan terdapat pada syntax nya.
\begin{enumerate}
\item Python 2 kita bisa menggunakan tanda kurung atau tidak, namun di Python 3 wajib menggunakan tanda kurung. Jika tidak maka akan terjadi error.
\item Python 2 ketika meminta suatu inputan, maka harus menggunakan \textbf{raw input} dan untuk Python 3 kita hanya perlu syntax \textbf{input} saja.
\end{enumerate}
\newpage
\section{Implementasi Python Pada Perusahaan di dunia}
\begin{enumerate}
\item Google : Pada mesin pencarian nya menggunakan bahasa python
\item Facebook : menggunakan Tornado, sebuah framework Python untuk menampilkan timeline
\item Instagram : menggunakan Django, framework Python sebagai mesin pengolah sisi server dari aplikasinya.
\item Dan masih banyak lagi.
\end{enumerate}
\end{document}