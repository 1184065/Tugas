\chapter{SIDANG PEKERJAAN PROYEK II}
\section{Tujuan}
Untuk	menguji	mahasiswa	atas	pekerjaan	Proyek II yang	 telah	dikerjakan	maka	diadakan	
sidang Proyek II.

\section{Waktu}
Sidang Proyek	diselenggarakan	pada	waktu	yang	telah	ditentukan	dan	diatur	pada	Petunjuk	
Pelaksana.

\section{Syarat	Sidang}
Proyek	yang	diajukan	ke	Sidang Proyek II adalah	Proyek	yang	telah	memenuhi	persyaratan	
sebagai	berikut	:

\begin{enumerate}
	\item Nilai	bimbingan	$ \geq80$ untuk	masing-masing	pembimbing .
	\item Aplikasi,	Sistem	atau	Alat	yang	dibuat	telah	selesai	$\geq90\%$ .
	\item Buku	laporan	telah	terselesaikan	100\% .
	\item Buku	 laporan	 telah	 diserahkan	 ke	 penguji	 paling	 lambat	 2	 minggu sebelum	 hari	
pelaksanaan	Sidang Proyek II .
	\item Diijinkan	sidang oleh	pembimbing,	dibuktikan	dengan	surat	permohonan	sidang Proyek
II .
	\item Telah	mendapat	nilai	Bimbingan	dari	pembimbing .
	\item 2	Minggu	sebelum	sidang	sudah	memberikan	laporan	ke	pembimbing,	jika	pembimbing	
berhalangan	segera	konfirmasikan	ke	koordinator	proyek.
	\item Sudah	melunasi	SPP.
	\item Mendaftarkan	sidang	ke	koordinator	proyek.
\end{enumerate}

\section{Penguji}
	\subsection{Penentuan Penguji}
		Penguji	ditentukan	oleh	Koordinator	Proyek	II	dan	penunjukannya	disampaikan	dengan	
diterbitkannya	surat	keputusan	tentang	pembentukan	panitia	sidang Proyek II.
	\subsection{Jumlah Penguji}
	Jumlah	penguji	adalah dua	orang. Yang	terdiri	dari	Ketua	Penguji	dan	anggota	penguji.
Ketua	Penguji		adalah	Pembimbing	Proyek II ,	dan	anggota	penguji	adalah	penguji	yang	
ditunjuk	 oleh	 koordinator	 yang	 kedua	 duanya	 disampaikan	 dengan	 surat	 keputusan.	
Sidang dengan	penguji	kurang	dari	2	orang	dianggap	batal.
	\subsection{Susunan	panitia	Sidang Proyek II}
	\begin{itemize}
		\item Ketua Penguji
		\item Anggota
	\end{itemize}
	
\section{Penilaian}
\begin{enumerate}
	\item Kriteria	penilaian	adalah	sesuai	dengan \textit{form} penilaian.
	\item Penilaian	menggunakan	angka	1-100 dengan	kategori	sebagai	berikut
		\begin{enumerate}[label=(\alph*)]
		\item \textit{Design	Antar	Muka}.
		\item Implementasi web	 service	 dengan	 Oauth	 2	 atau	 				  sistem	 token	 antara	\textit{Client/Servernya}.
		\item Bahasa	Pemograman	menggunakan	framework.
		\item Seluruh	 table	 diisi	 dengan	 data	 dummy	 sebanyak	 satu	 juta	 record	 masing-masing	 ketika	demo	 sidang, database	juga	menggunakan	procedure,	 fungsi	dan atau trigger. Implementasikan	juga \textbf{redis} sebagai	database	cache
		\item Study Kasus.
		\end{enumerate}
\end{enumerate}

\section{Kehadiran Pembimbing}
	Sidang tanpa	kehadiran	pembimbing	dianggap	BATAL	dan	pelaksanaannya	akan	ditentukan	
kemudian.

\section{Persyaratan	Administrasi	Sidang}
	Sebelum	sidang harus	tersedia \textit{form-form} yang	diperlukan	yaitu	:
	\begin{enumerate}
		\item Berita	Acara	Sidang
		\item Lembar	catatan/perbaikan	Proyek	
		\item Lembar	penilaian	sidang Proyek
		\item Lembar	persyaratan	untuk	lulus	bersyarat	
		\item Lembar	penilaian	bimbingan	
	\end{enumerate}
	\par \textit{Form} tersebut	 diatas	 harus	 sudah	 ada	 pada	 peguji	 sidang	 proyek	 sebelum	 sidang	 itu	
dimulai.

\section{Status	Hasil	Sidang Proyek}
Status	hasil	sidang Proyek	adalah	sebagai	berikut	:
\begin{enumerate}
\item Lulus.
\item Lulus Bersyarat.
\item Tidak Lulus.
\end{enumerate}
\par Status	tersebut	ditentukan	oleh	sidang	dewan	penguji