\chapter*{Perusahaan Dunia yang menggunakan bahasa pemrograman Python}

\section*{\textit{Spotify}}
\par
\textit{Spotify} merupakan layanan musik streaming yang sudah banyak digunakan di seluruh dunia. System rekomendasi spotify bergantung pada analisis data yang sangat besar untuk menginterpretasikan analais tersebut Spotify menggunakan luigi, modul python yang singkron dengan hadoop. Modul open source ini menangani satu library dengan library lain nya agar saling bekerjasama, dan mengkolosidasi eror log secara cepat

\section*{\textit{Netflix}}
\par
\textit{Netflix} .Pada netfilx bahasa pemprograman yang digunakan adalah bahasa pemprograman python, bahasa pemprograman ini digunakan pada Central Alert Gateway yang akan me-reroute alert dan mengirimkannya pada individu yang akan melihatnya serta juga  dapat secara otomatis reboot atau menghentikan proses yang dianggap bermasalah. Selain itu python juga digunakan untuk menelusuri riwayat dan perubahan pengaturan keamanan.

\section*{\textit{Google}}
\par
\textit{Google} ini sudah menggunakan bahasa pemprograman python ini sudah sajak dari awal berdirinya. Dan pada saat ini bahasa pemprograman python merupakan salah satu bahasa pemprograman server-side resmi di google. Meskipun ada script yang ditulis untuk google menggunakan bahasa perl dan bash, maka nantinya script tersebut akan diubah ke python terlebih dahulu, karena kemudahan dalam perawatannya.
\section*{\textit{Instagram}}
\par
\textit{Instagram}. Pada instagram ini menggunakan bahasa pemprograman python dalam task queuennya atau fitur dimana setiap pengguna dapat berbagi foto atau video ke beberapa social network lainnya  

\section*{\textit{Industrial Light and Magic}}
\textit{Indusrial Light and Magic} adalah Studio spesial-efek yang digunakan pada pemutaran efek di film Star Wars. Dalam pembuatan efek ledakan ILM menggunakan Python dikarenakan dapat menghemat waktu dalam pembuatan efek tersebut.   
\par

