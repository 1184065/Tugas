\chapter*{Perusahaan Dunia yang menggunakan bahasa pemrograman Python}

\section*{\textit{Spotify}}
\par
\textit{Spotify} merupakan layanan musik streaming yang sudah banyak digunakan di seluruh dunia. Dalam bidang menganalisis data spotify menggunakan Bahasa Pemrograman Python, dalam pengimplementasiannya Tim Spotify menggunakan Luigi, modul yang ada di Python yang disingkronisasikan pada sebuah software yang memudahkan programmer membuat aplikas web atau disebut framework yang berbasis Java yang memungkinkan pemrosesan data dalam waktu cepat.Penerapan bahasa Python juga digunakan dalam penerapan fitur Radio dan Discover serta fitur merekomendasikan orang yang mungkin akan diikuti.

\section*{\textit{Netflix}}
\par
\textit{Netflix} merupakan layanan pemutaran film atau tayangan yang memungkinkan para penggunaknya menggunakan di manapun dan kapanpun. Netflix menggunakan bahasa pemrograman Python. Penggunaan Python di Netflix terdapat pada Central Alert Gateaway (C.A.G) ini akan me-reroute alert dan mengirimkannya pada kelompok atau individu yang dapat melihatnya dan secara otomatis reboot atau menghentikan proses yang dianggap bermasalah dan digunakan untuk menulusuri riwayat dan perubahan pengaturan keamanan. Tetapi sama halnya seperti spotify, Netflix menggunakan python untuk menganilisis data dan lebih utama terlihat pada bagian bagaimana netflix merekomendasikan film kepada pelangganya.

\section*{\textit{Pinterest}}
\par
\textit{Pinterest} adalah aplikasi web yang digunakan untuk mengumpulkan hal-hal yang menarik berdasarkan kriteria tertentu yang sering dikunjungi di jaman sekarang. Pinterest menggunakan Bahasa Pemrograman Python dari awal mereka membangunnya itulah sebabnya bookmarking (Sebuah metode bagi pengguna internet untuk mengorganisasi, menyimpan, mengelola, dan mecari penanda sumber daya yang tersedia secara online) yang ada di pinterest begitu terstruktur dan mudah untuk diatur.

\section*{\textit{Instagram}}
\par
\textit{Instagram} merupakan sebuah aplikasi berbagi foto dan video secara digital yang digunakan oleh lebih dari 400 juta user yang aktif setiap harinya. Instagram menggunakan bahasa pemrograman python dalam task queuenya atau fitur dimana pada saat yang bersamaan instagram dapat melakukan posting ke beberapa social network lainnya seperti Facebook, Twitter, dll. 

\section*{\textit{Industrial Light and Magic}}
\textit{Indusrial Light and Magic} adalah Studio spesial-efek yang digunakan pada pemutaran efek di film Star Wars. Dalam pembuatan efek ledakan ILM menggunakan Python dikarenakan dapat menghemat waktu dalam pembuatan efek tersebut.   
\par

