\chapter*{ SEJARAH PYTHON}

\par
	Python diciptakan atau ditemukan oleh Guido van Rossum di Centrum Wiskunde \& Informatica (CWI) di Belanda pada tahun 1990-an. Bahasa Pemrograman Python terinspirasi dari bahasa pemrograman ABC. Pemberian nama Phyton sendiri bukan berati ular yang banyak kita ketahui tetapi Guido adalah seorang penggemar grup komedian Inggris bernama Monthy Pyton, maka dari itu ia memberi nama Bahasa Pemrograman yang ia buat dengan nama Pyton. Sampai sekarang Guido van Rossum masih menjadi penulis utama dalam bahasa pemrograman Phyton ini walaupun sudah ribuan orang ikut berkontribusi dalam penulisan phyton dan mengembangkannya. Phyton merupakan bahasa pemrograman yang bersifat open source multiguna yang berfokus pada keterbacaan kode. Phyton diklaim merupakan bahasa pemrograman yang menggabungkan kemampuan dengan sintaksis kode yang sangat jelas dilengkapi dengan fungsionalitas yang besar serta mampu menerima dengan baik. Pada tahun 1995, Guido melanjutkan pembuatan phyton di Corporation for National Research Intiative (CNRI) di Virginia Amerika, dan merilis beberapa versi terbaru dari phyton. Hampir semua versi Phyton rilis menggunakan lisensi GFL-comatible. Dibawah ini merupakan rilisan resmi versi mayor dan minor phyton:
		
\begin{enumerate}
\item Pyton 1.0 – 1.6 Rilis dari tahun 1994- pertengahan 2000
\item Pyton 2.0 – 2.7 Rilis dari tahun pertengahan 2000-2010 
\item Pyton 3.0 – 3.7 Rilis dari tahun 2008-2018
\end{enumerate}

\par
	Pyton yang banyak digunakan sekarang adalah Pyton 2 dan Pyton 3, disetiap pengembangan versi tentu saja memiliki peningkatan kualitas seperti yang terjadi di Pyton 2 dan Pyton 3. Ketika membuat kodingan di Phyton2 dan di complide di shell pyton3 akan terjadi eror karena script phyton 2 sudah tidak compatibel di shell phyton 3, karena di phyton 3 memerlukan tanda kurung () sedangkan di pyton 2 tidak memerlukannya. Di Pyton 3 akan terlihat lebih rapih dibandingan di pyton 2 
