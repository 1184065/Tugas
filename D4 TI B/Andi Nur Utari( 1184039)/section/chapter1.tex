\chapter{Teori Mengenal Python dan Anaconda}
\section{Sejarah Python}
Python adalah bahasa pemrograman tingkat tinggi yang dirilis oleh Guido Van Rossum di Centrum Wiskunde and Infomatic (CWI) di Belanda pada tahun 1991. Bahasa Python terinspirasi dari bahasa pemrograman ABC. Nama Python berasal dari acara televisi humor favorite Guido yang berjudul Monty Python's FLying Circus.
\\ Python 1.0 dirilis pada tahun 1994. Versi python 1.2 dikeluarkan oleh CWI pada tahun 1995. Guido pindah ke CNRI sambil melanjutkan pengembangan Python. Versi 1.6 dikeluarkan tahun 2000. Pada mei 2000, Guido dan tim pindah ke BeOpen.com dan membentuk tim BeOpen Python Labs. Pada oktober tahun 2000, tim pindah ke Digital Creation (perusahaan Zope). Pada tahun 2001, dibentuk Python Software FOundation (PSF) merupakan organisasi khusus untuk semua hal yg berkaitan dengan hak intelektual Python. Pada tahun 2000 dirilis Python 2.0 Python 3.0 dirilis pada tahun 2008. Versi terakhir dari python adalah 3.7 yang dirilis Juni 2018. Semua versi dari Python dirilis dengan bersifat open source dimana layanannya bisa digunakan secara gratis tanpa adanya batasan untuk penggunanya dan bisa dikembangkan oleh siapa saja.

\section{Perbedaan Python 2 dan 3}

Python 2 dan 3 memiliki perbedaan antar versinya. Karena tiap versi akan terus memperbaiki dan mengoptimalkan fitur-fitur yang ada sebelumnya untuk mewujudkan produk yang efektif. 
\begin{enumerate}

\item Syntax untuk mencetak (print)
Pada python 2 print diperlakukan seperti statement, bias menggunakan tanda kurung dan juga bias tidak menggunakan tanda kurung untuk mencetak.
Contoh : \\
Print “Nama saya Tari” \\
Print (“Nama saya Tari”) \\
Sedangkan pada python 3 print diperlakukan sebagai function.
Contoh  \\
Print (“Saya anak pertama”) 

\item Syntax untuk inputan \\
Pada Python 2 menggunakan syntax inputan raw input dan untuk pemanggilan variabelnya tanpa menggunakan tutup kurung (). \\
nama = raw input ("Masukkan NPM anda")\\
print nama  \\
Sedangkan pada python 3 tanpa mengguanakan raw lebih mudah untuk melakukan pemanggilan variable, contohnya \\
Nama = input("Masukkan NPM anda") \\
print (nama) 

\item Pembagian pada integer
Pada python 2 semua tipe data angka yang tidak desimal akan diperlakukan seperti integer. seperti 3/2 = 1.5 Python 2 menggunakan floor division dimana dibulatkan ke nilai paling rendah seperti. Sedangkan pada python 3 pembagian pada bilangan integer lebih intuitif. untuk mendapatkan hasil desimal maka pada pembagiannya ditambahkan jadi seperti ini 3.0/2.0 = 1.5 untuk mendapatkan floor division maka digunakan //. 

\item Dukungan unicode
Python 2 menggunakan alfabet ASCII dimana masih terbatas pada beberapa ratus karakter. untuk menggukan unicode yang bisa mendukung lebih banyak karakter maka harus mengetik u"Halo!" dengan tambahan u didipannya.
Python 3 menggunakan unicode secara default. yang memudahkan untuk menghemat waktu dan mudah diisikan dan ditampilkan. Karena unicode pada python 3 mendukung berbagai karakter linguistik termasuk menampilkan emoji.

\end{enumerate}
\section{Implementasi dan Penggunaan Python di perusahaan dunia.}
\subsection{Implementasi}
Bahasa pemrograman Python dikenal mudah dipelajari karena struktur sintaknya rapi dan mudah dipahami. 
\begin{enumerate}
\item 1. Pada dunia kesehatan 
\item 2. Pemerintahan
\item 3. Keamanan Negara
\item 4. Ilmu pendidikan
\item 5. Internet
\item 6. Perusahaan / Bisnis
\end{enumerate}

\subsection{Penggunaan Python}
Sampai saat ini bahasa Python banyak di gunakan diberbagai perusahaan besar di dunia. diantaranya yaitu
\begin{enumerate}
\item Google merupakan salah satu perusahaan besar didunia yang menggunakan bahasa python sejak awal berdirinya. Python adalah bahasa pemrograman yang sangat penting bagi Google. Google pernah merekrut Guido Van Rossum untuk bekerja di perusahaannya. Google menggunakan python sebisa mungkin walaupun sudah ada script dalam bahasa Perl dan Bash, namun script itu nanti akan diubah ke bahasa Python untuk kemudahan perawatannya.
\item Intagram adalah aplikasi yang banyak digunakan saat ini untuk membagikan media berupa foto atau video. Bagi pera pengembang di instagram bahasa python sangat rama pengguna , sederhana, dan rapi. Dan bahasa python sangatlah populer sehingga mereka tidak akan sulat mencari pengembang baru untuk memperbesar tim pengembangannya.
\item Sportify adalah penyedia layanan musik streaming, perusahaan sportify menggunakan bahasa python untuk analisis data dan backend, dimana backend sportify banyak service yang berkomunikasi lewat OMQ (Zero OMQ) merupakan framework dan library open source untuk networking. Sportify menggunakan python karena menyukai kecepatan pipeline developement.
\item Netflix, pada Central Alert Gateway di aplikasi netflix digunakan Python. Pada aplikasi RESTful ini akan mereroute alert dan memngirimkannya pada pengguna yang berhak melihatnya. Aplikasi ini juga otomatis reboot atau berhenti pada proses yang dianggap bermasalah. Pyhton juga digunakan pada aplikasi menelusuri riwayat dan perubahan keamanan pada netflix.
\item Industrial Light and Magic adalah studio spesial efek. dengan menggunakan python ILM dengan mudah membungkus komponen software dan mudah untuk meningkatkan aplikasi grafis.
\item Youtube merupakan perusahaan situs video yang terbesar didunia. Pyhton digunakan untuk sebagian besar kode pada youtube.
\item Facebook menggunakan framework python yaitu tornado untuk menampilkan timeline\
\item Pinterest, aplikasi yang banyak menggunakan Python untuk pembagunan aplikasi tersebut.
\item NASA yaitu badan antariksa amerika banyak menggunakan bahasa pemrograman python pada bidan sainsya.
\end{enumerate}
